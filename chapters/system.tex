\chapter{DCE in a Periodic Potential} %\label{ch:system}


\section{Circuit analysis for a 1-D periodic potential quantum network}\label{sec:circ_an}
We want to study the DCE in a superconducting circuit as in (Johansson). We characterize the circuit in terms of dynamical fluxes $\Phi_i$. Assuming symmetric SQUIDs, the Lagrangian for this system is
\begin{equation}
\begin{split}
&\mathcal{L}=\frac{1}{2}\sum_{i=1}^{\infty}\left(\Delta x C_{0} \left(\dot{\Phi}_{i}\right)^{2} - 
\frac{\left(\Phi_{i+1}-\Phi_{i}\right)^{2}}{\Delta x L_{0}}\right)\\
&+ \sum_{s=1}^{N}\left[\frac{1}{2}C_{J, s}\left(\dot{\Phi}_{s}\right)^{2}+\left(\frac{2\pi}{\Phi_{0}}\right)^{2} E_{J, s}\left(\Phi_{\text{ext}}\right) \cos\left(2\pi \frac{\Phi_{s}}{\Phi_0}\right)\right].
\end{split}
\end{equation}
Where $C_0, L_0$ are the characteristic capacitance and inductance (per unit length) of the CPW, $\Phi_0=\frac{h}{2e}$ is the flux quantum, and $\Phi_{J,s}$ indicate the flux field at SQUID sites $x_s$. The SQUIDs are characterized by Josephson Capacitance $C_{J,s}$ and the Josephson energy $E_J$. The energy of the SQUID is
\begin{equation}
    E_{J,s} (\Phi_{\text{ext},s}) = 2 \left\vert \cos\left(\pi \frac{\Phi_{\text{ext},s}}{\Phi_0}\right)\right\vert .
\end{equation}
We impose the requirement that all SQUIDs in our system are equal, that is $C_{J,s} = C_J$, $E_{J,s}(\Phi_{\text{ext},s}) = E_J(\Phi_{\text{ext}}) $ for all $s$.
If we assume the plasma frequency of the SQUIDs is far exceeding the characteristic frequencies of the circuit $\Phi_J/\Phi_0 \ll 1$, and that SQUIDs are operated in the phase regime where $E_J \gg \frac{(2e)^2}{2C_J}$. 
\begin{equation}
\mathcal{L}=\frac{1}{2}\sum_{i=1}^{\infty}\left(\Delta x C_{0} \left(\dot{\Phi}_{i}\right)^{2} - 
\frac{\left(\Phi_{i+1}-\Phi_{i}\right)^{2}}{\Delta x L_{0}}\right) 
 + \frac{1}{2} \sum_{s=1}^{N}\left(C_J\left(\dot{\Phi}_{s}\right)^{2}-\left(\frac{2\pi}{\Phi_{0}}\right)^{2} E_J\left(\Phi_{\text{ext}}\right) \Phi_{s}^{2}\right).
\end{equation}
Following the canonical quantization procedure, we now find the Hamiltonian of the system by performing a Legendre transformation $\mathcal{H}=\sum_{i}\frac{\partial\mathcal{L}}{\partial\dot{\Phi}} \dot{\Phi} - \mathcal{L}$.
\begin{equation}
\mathcal{H}=\frac{1}{2}\sum_{i=1}^{\infty}\left[\frac{P_i^2}{\Delta x C_{0}} -\left(\frac{\left(\Phi_{i+1}-\Phi_{i}\right)^{2}}{\Delta x L_{0}}\right)\right] 
 + \frac{1}{2} \sum_{s=1}^{N}\left(\frac{P_s^2}{C_J}+\left(\frac{2\pi}{\Phi_{0}}\right)^{2} E_J\left(\Phi_{\text{ext}}\right) \Phi_{s}^{2}\right).
\end{equation}

We obtain the dynamics of the system by using the Heisenberg equations of motion for $P_i$ and $P_s$. In the continuous limit $(\Delta x \to 0)$, the flux field satisfies the 1-dimensional massless Klein-Gordon equation.
\begin{equation}
\frac{\partial^2\Phi(x,t)}{\partial t^2} - \nu^2 \frac{\partial^2\Phi(x,t)}{\partial x^2} =0
\end{equation}

where we define $\nu = \hbar / C_0 L_0$ as the propagation speed in the CPW.

The Heisenberg equation for $P_s$ (in continuous limit) provides the boundary condition at the SQUID sites $x_s$.
\begin{equation}\label{eq:BC_field}
C_{J} \ddot{\Phi}_{s}+\hbar\left(\frac{2 \pi}{\Phi_{0}}\right)^{2} E_{j}\left(\Phi_{\text{ext}}\right) \Phi_{s} -\frac{\hbar}{L_{0}}\left(\left.\frac{\partial \Phi_{s}}{\partial x}\right|_{x_s^{+}}-\left.\frac{\partial \Phi_{s}}{\partial x}\right|_{x_s^{-}}\right)=0.
\end{equation}

\section{Periodic potential and Kronig-Penney model}\label{sec:Kronig-Penney}
%
\noindent
Let us first consider the background periodic potential in the static case where we apply a DC drive so that the SQUID energy $E_J(\Phi_{\text{ext}}) = E_J^0$ is constant. Following the canonical quantization procedure, 
\color{blue} we turn the classical flux field ${\Phi}_n(x,t)$ 
in compartment $n$ of the periodic array \color{red} (see Fig.\,X) \color{blue}
into a field operator $\hat{\Phi}_n(x,t)$, which is expanded in terms of 
annihilation and creation operators $\color{blue} \hat{a}_n(k)$ and 
$\color{blue} {\hat a}_n^\dagger(k)$:
%
\begin{equation} \label{eq:flux_field_orig}
    \hat{\Phi}_n(x,t) = \sqrt{\frac{\hbar}{2 C_0}} 
    \int_{-\infty}^{\infty}\frac{dk}{2 \pi} \frac{1}{\sqrt{\omega_k}}
    \left\{ \hat{a}_n(k) \psi_k(x)e^{-i \omega(k) \, t} + 
    \hat{a}_n^{\dagger}(k) \psi_k^*(x) e^{i \omega(k) \, t} \right\} \, \, .
\end{equation}
%
The annihilation and creation operators obey the commutation relations
%
\begin{equation} \label{eq:cra_orig}
    \left[ \hat{a}_n(k),{\hat a}_{n'}^\dagger(k') \right] = \delta_{nn'} \, 2 \pi \delta(k - k')
\end{equation}
%
and thus have units of $\text{length}^{1/2}$.
Equation (\ref{eq:flux_field_orig}) also incorporates the Bloch-type functions 
%
\begin{equation} \label{eq:psi1}
\psi_k(x) = e^{i k x} u_k(x) \, \, ,   
\end{equation}
%
where $u_k(x)$ has period $\ell$, i.e., $u_k(x + \ell n) = u_k(x)$ for all integers $n$.
The Bloch functions $\psi_k(x)$ in Eq.\,(\ref{eq:psi1}) are unitless, orthogonal with respect to 
the 1D wave vector $k$, and normalized such that
%
\begin{equation} \label{eq:psi1_norm_orig}
\int_{-\infty}^{\infty} dx \, \psi^*_k(x) \psi_{k'}(x) = 2 \pi \, \delta(k - k') \, \, .
\end{equation}
%
The prefactor $\displaystyle{\sqrt{\frac{\hbar}{2 C_0}}}$ in Eq.\,(\ref{eq:flux_field_orig}) is chosen 
to obtain the canonical commutation relation 
%
\begin{equation} \label{eq:commrelphi}
\left[ \hat{\Phi}_n(x,t), \hat{P}_n(x',t) \right] = i \hbar \delta(x-x')
\end{equation}
%
where $\hat{P}_n(x,t) = C_0 \displaystyle{\frac{d}{dt}} \hat{\Phi}_n(x,t)$ is the 
canonical conjugate of $\hat{\Phi}_n(x,t)$ and $C_0$ is the characteristic capacitance of the CPW
per unit length. 

We now rewrite Eq.\,(\ref{eq:flux_field_orig}) in terms of unitless variables 
by expressing lenghts in units of $\ell$ and times in units of $\ell/v$
as outlined in \color{red} Section X:  \color{blue}  
%
\begin{equation} \label{eq:flux_field}
    \hat{\phi}_n(x,t) = 
    \int_{-\infty}^{\infty}\frac{dk}{2 \pi} \frac{1}{\sqrt{\omega_k}}
    \left[ \hat{a}_n(k) \psi_k(x)e^{-i \omega(k) \, t} + 
    \hat{a}_n^{\dagger}(k) \psi_k^*(x) e^{i \omega(k) \, t} \right]
\end{equation}
%
with the unitless field operator in compartment $n$ of the periodic array
%
\begin{equation} \label{eq:ufo}
\hat{\phi}_n(x,t) := \sqrt{\frac{2 C_0 v}{\hbar}} \, \hat{\Phi}_n(x,t) \, \, .
\end{equation}
%
Unitless annihilation and creation operators are defined by $\hat{a}_n(k)/\sqrt{\ell}$ 
and ${\hat a}_{n}^\dagger(k) / \sqrt{\ell}$ and again denoted by
$\hat{a}_n(k)$, ${\hat a}_{n}^\dagger(k)$ to keep the notation simple. 
They obey the commutation relations 
%
\begin{equation} \label{eq:cra}
    \left[ \hat{a}_n(k),{\hat a}_{n'}^\dagger(k') \right] = \delta_{nn'} \, 2 \pi \delta(k - k') \, \, .
\end{equation}
%
All variables in Eq.\,(\ref{eq:flux_field}) are unitless by expressing lenghts in units of $\ell$
and times in units of $\ell/v$.


\color{black}


Given the time dependence of the flux field $\Phi(x,t)$, we can write our boundary condition (\ref{eq:BC_field}) as
\begin{equation}
\left[-\omega^2 C_{J}+\hbar\left(\frac{2 \pi}{\Phi_{0}}\right)^{2} E_{j}^0\right]\Phi_{s} -\frac{\hbar}{L_{0}}\left(\left.\frac{\partial \Phi_{s}}{\partial x}\right|_{x_s^{+}}-\left.\frac{\partial \Phi_{s}}{\partial x}\right|_{x_s^{-}}\right)=0.
\end{equation}

By construction, we require $\psi_k(x)$ to satisfy the same eigenvalue problem
\begin{equation}\label{eq:BC_bloch}
\left(\left.\frac{\partial}{\partial x}\right|_{x_s^{+}}-\left.\frac{\partial}{\partial x}\right|_{x_s^{-}}\right)\psi_{k}(x)=\lambda \psi_k(x),
\end{equation}

where $\lambda$ is defined as:
\begin{equation}
\lambda = \left(\frac{L_0}{\hbar}\right)  
\left[-\omega^2 C_{J}+\hbar\left(\frac{2 \pi}{\Phi_{0}}\right)^{2} E_{j}^0\right]
\end{equation}

as well as the continuity condition
\begin{equation}\label{eq:continuity_bloch}
\biggl.\psi_{k}(x)\biggl|_{x=x_{s}}=\biggl.\psi_{k}(x)\biggr|_{x=x_{s}^{+}}.
\end{equation}

From here on, we work with units where $\tilde{x}=x/\ell$ , $\tilde{\omega}=\omega\ell/\nu_{\text{CPW}}$, $\tilde{k}= k\ell$. We express solutions as
\begin{align}
\psi_{k,n}(x) = A_n e^{i\tilde{\omega}(\tilde{x}-n)} + B_n e^{-i\tilde{\omega}(\tilde{x}-n)},\hspace{16ex} n-1 < \tilde{x} < n,\\
\psi_{k,n+1}(x) = A_{n+1} e^{i\tilde{\omega}(\tilde{x}-n-1)} + B_{n+1} e^{-i\tilde{\omega}(\tilde{x}-n-1)},\hspace{10ex} n < \tilde{x} < n+1.
\end{align}

We follow a standard procedure to find wavefunction solutions to the Kronig Penney model, employing a transfer matrix method to find solutions for $A_{n+1}, B_{n+1}$ in terms of $A_n, B_n$. Using (\ref{eq:continuity_bloch}), we can express (\ref{eq:BC_bloch}), with a transfer matrix 
\begin{equation}
 \begin{pmatrix}
 A_{n+1} \\ B_{n+1}
 \end{pmatrix}
 = T 
 \begin{pmatrix}
 A_n \\ B_n
 \end{pmatrix}
\end{equation}{}

We construct matrix $T=PL^{-1}VL$ by defining the following matrices 
\begin{align}
L = \begin{pmatrix}
1 & 1 \\
i\tilde{\omega} & -i\tilde{\omega}
\end{pmatrix} \\
V = \begin{pmatrix}
1 & 0 \\
\lambda & 1
\end{pmatrix} \\
P = \begin{pmatrix}
e^{i\tilde{\omega}} & 0 \\
0 & e^{-i\tilde{\omega}}.
\end{pmatrix}
\end{align} 

We then find eigenvalues for $T$
\begin{equation}
    \mu_{(1/2)} = \cos{\tilde{\omega}} + \frac{\lambda}{2\tilde{\omega}}\sin{\tilde{\omega}}\mp i \sqrt{1-\left(\cos{\tilde{\omega} + \frac{\lambda}{2\tilde{\omega}}\sin{\tilde{\omega}}}\right)^2}
\end{equation}

The determinant of $T$ is 1 (det $T = 1$), thus the eigenvalues fulfill $\mu_1 \cdot \mu_2$. We find that $|\mu_1| = |\mu_2|=1$, and thus we can define $\tilde{k}$ as $e^{\pm i\tilde{k}} = \mu_{(1/2)}$, which leads to the relation
\begin{equation}
    \cos{\tilde{k}} = \cos{\tilde{\omega}} + \frac{\lambda}{2\tilde{\omega}}\sin{\tilde{\omega}}
\end{equation}

We find a band structure, where physical states correspond to solutions such that
\begin{equation}\label{eq:band_condition}
    \left|\cos{\tilde{\omega}} + \frac{\lambda}{2\tilde{\omega}}\sin{\tilde{\omega}}\right|\leq 1.
\end{equation}

The (normalized) eigenvectors of $T$ are given by:
\begin{gather}
    \hat{V}^{(1)}
    = N
    \begin{pmatrix}
    e^{i\tilde{\omega}}\left\lbrace-\cos{\tilde{\omega}} + \frac{\lambda}{2\tilde{\omega}}\sin{\tilde{\omega}} - \frac{2\tilde{\omega}}{\lambda}\sqrt{1-\left(\cos{\tilde{\omega}+\frac{\lambda}{2\tilde{\omega}}\sin{\tilde{\omega}}}\right)^2}
    \right\rbrace \\ 1 \end{pmatrix} \\
    \hat{V}^{(2)}
    = N
    \begin{pmatrix}
    \left\lbrace-\cos{\tilde{\omega}} + \frac{\lambda}{2\tilde{\omega}}\sin{\tilde{\omega}} + \frac{2\tilde{\omega}}{\lambda}\sqrt{1-\left(\cos{\tilde{\omega}+\frac{\lambda}{2\tilde{\omega}}\sin{\tilde{\omega}}}\right)^2}
    \right\rbrace \\ e^{-i\tilde{\omega}} \end{pmatrix}
\end{gather}

where $N$ is a normalization constant. From these eigenvectors, we can obtain the states of the system at the $n$th lattice site by plugging in a state $\psi_{k,n=0}(x)$ and using the eigenvalue equation for the transfer matrix $T$.
\begin{equation}
    \begin{pmatrix}
    A_n^{(1/2)} \\ B_n^{(1/2)}
    \end{pmatrix}
    = e^{\pm i\tilde{k}n} 
    \begin{pmatrix}
    A_0^{(1/2)} \\ B_0^{(1/2)}
    \end{pmatrix}
\end{equation}

Thus, we arrive at functions $\psi_{k}(x)$, which we can write in the Bloch form
\begin{equation}\label{eq:bloch_waves}
    \psi_{\tilde{k}}^R (x) =e^{i\tilde{k}\tilde{x}} \hat{u}_{\tilde{k}}^{(1)}(\tilde{x}), \hspace{12pt}  \psi_{\tilde{k}}^L(x) =e^{-i\tilde{k}\tilde{x}} \hat{u}_{\tilde{k}}^{(2)}(\tilde{x}) 
\end{equation}

Where $\hat{u}_{\tilde{k}}^{(1/2)}$ are the normalized forms of periodically repeated functions
\begin{gather}
      u_{\tilde{k}}^{(1)}(\tilde{x}) = 
      \frac{A_0^{(1)} e^{i(\tilde{\omega}-\tilde{k})(\tilde{x}-1)} + B_0^{(1)} e^{-i(\tilde{\omega}+\tilde{k})(\tilde{x}-1)}}{A_0^{(1)}+B_0^{(1)}}\\
       u_{\tilde{k}}^{(2)}(\tilde{x}) =
      \frac{A_0^{(2)} e^{i(\tilde{\omega}+\tilde{k})(\tilde{x}-1)} + B_0^{(2)} e^{-i(\tilde{\omega}-\tilde{k})(\tilde{x}-1)}}{A_0^{(2)}+B_0^{(2)}}
\end{gather}

normalized such that
\begin{equation}
\int_0^1 \hat{u}_{\tilde{k}} (\hat{u}_{\tilde{k}})^* d\tilde{x} = 1
\end{equation}

Since the functions \ref{eq:bloch_waves} form a complete set, we can use them to expand our flux field as in \ref{eq:flux_field}.

\section{Static Flux Solution}\label{eq:Static_Flux}

We proceed to obtain solutions for the problem with applied static magnetic flux. Taking the Fourier transform of the field \ref{eq:flux_field}, expanded in terms of the bloch waves
\begin{equation}
    \Phi(x, \mu = \int_{-\infty}^\infty \Phi(x,t) e^{i\mu t} dt
\end{equation}

We obtain the following forms for the field:
\begin{gather}
    \Phi(x,\mu) = \sqrt{\frac{\hbar}{2 C_0}} \left| \left.\frac{d\omega}{dk}\right|_{\omega=\mu}\right|^{-1}\frac{1}{\sqrt{\mu}}\biggl\lbrace a_R(K) \psi^R_K(x) + a_L(K)\psi_K^L(x)\biggr\rbrace, \hspace{12pt} \mu > 0.\\
    \Phi(x,\mu) = \sqrt{\frac{\hbar}{2 C_0}} \left| \left.\frac{d\omega}{dk}\right|_{\omega=\mu}\right|^{-1}\frac{1}{\sqrt{|\mu|}}\biggl\lbrace a_R^\dagger(K) \left[\psi^R_K(x)\right]^* + a_L^\dagger(K)\left[\psi_K^L(x)\right]^*\biggr\rbrace, \hspace{12pt} \mu < 0.
\end{gather}

Where we have used (\ref{eq:bloch_waves}), defining $K = |k|$. As well as 
\begin{equation}
    a(K) = 
    \begin{cases}
    a_R(K) &\text{if k > 0,}\\
    a_L(K) &\text{if k < 0.}
    \end{cases}
\end{equation}




Which we input into the  boundary condition \ref{eq:BC_field} to obtain

\begin{equation}
    \left[a_R(K) - b_R(K)\right]\mathcal{A}_K + \left[a_L(K) - b_L(K)\right]\mathcal{A}_K^* = 0 
\end{equation}

where we define
\begin{gather}
    \mathcal{C}_K = \left.u_K(x)\right|_{x=x_s}\\
    \mathcal{D}_K = \left.u_K^{\prime}(x)\right|_{x=x_s}\\
    \mathcal{A}_K = ik\mathcal{C}_K + \mathcal{D}_K
\end{gather}

