
%A fundamental consequence of quantum field theory is the reformulation of vacuum as full of virtual fluctuations. The dynamical Casimir %effect (DCE) is a consequence of this formulation where the modulation of a mirror results in real photons being created by a perturbative %scattering process in the quantum electromagnetic field. DCE radiation can be generated in a superconducting quantum circuit, with %superconducting quantum interference devices (SQUIDs) acting as mirrors modulated by an applied magnetic field. Using the concepts of %periodic lattices from solid state physics and photonic crystals, we propose a SQUID periodic lattice architecture. We develop a mathematical %model to describe the dynamics of the system, calculate the output photon-flux density, and discuss the possibility of engineering the band %structure of the radiated light with possible applications in quantum information technologies.

The dynamical Casimir effect (DCE) is the generation of real photons out of the quantum vacuum due to a rapid modulation of boundary conditions for the electromagnetic field, such as a mirror oscillating at speeds comparable to the speed of light. 
Previous work demonstrated experimentally that
DCE radiation can be generated 
in electrical circuits based on superconducting microfabricated waveguides, where a rapid modulation of boundary conditions corresponding to semi-transparent mirrors is realized by tuning the applied magnetic flux through superconducting quantum-interference devices (SQUIDs) that are embedded in the waveguide circuits. We propose a novel SQUID periodic lattice architecture, in which SQUIDs embedded in a coplanar waveguide (CPW) form the sites of a one-dimensional periodic lattice, resulting in a band structure and band gaps for the DCE radiation akin to classical photonic crystals. The band structure in our "quantum photonic crystals"
can be tuned by the spatial distance between SQUIDs in the lattice and their Josephson energy. 
Moreover, the harmonic drive of the SQUIDs generating the DCE radiation can be tuned in terms of the drive frequency, amplitude, and phase.  
The latter two parameters can be modulated for each SQUID in the periodic array individually, making our proposed lattice architecture quite versatile. We find a rich interplay between the band structure of the lattice, 
the harmonic drive of the SQUIDs, and the DCE photon-flux density, which thus allows us to control, guide, and manipulate DCE radiation. 
We develop a theoretical and computational model for our proposed system and calculate the DCE radiation 
for various experimental setups. In particular, we show that a harmonic drive that breaks the left-right symmetry
results in quasi unidirectional DCE radiation. Possible applications of our results in quantum information technologies
are discussed.








