 \chapter{DCE in a Periodic Potential} \label{ch:system}


\section{Circuit analysis for a 1D periodic potential quantum network}\label{sec:circ_an}

%%%%%%%%%%%%%%%%%%%%%%%%%%%%%%%%%%%%%%%%%%%%%%%%%%%%%%%%%%%%%%

\subsection{Lagrangian formulation}
\noindent
We now proceed to the study of the DCE in our proposed system, which consists of a periodic array of CPWs connected by SQUIDs (see Figure \ref{fig:circuit_diagram}). To define the Lagrangian we discretize the system into 
segments $i$ of length $\Delta x$ with capacitance $\Delta x \, C_0$, inductance $\Delta x \, L_0$, and dynamical fluxes $\Phi_i(t)$ 
(see \cite{Vool2017}).
Assuming symmetric SQUIDs, the Lagrangian density for this system is 
%
\begin{equation} \label{eq:lagn1}
\begin{split}
\mathcal{L} = \, & \frac{1}{2} \sum_i \left( \Delta x \, C_{0} \left(\dot{\Phi}_{i}\right)^{2} \, - \, 
\frac{\left(\Phi_{i+1}-\Phi_{i}\right)^{2}}{\Delta x \, L_{0}} \right)  \\[2mm]
& + \sum_n \left[ \frac{1}{2} \, C_{J,\,n} \left(\dot{\Phi}_{J,\,n} \right)^{2} \, + \, 
E_{J,\,n}(t) \cos\left(2\pi \frac{\Phi_{J,\,n}}{\Phi_0} \right) \right]
\end{split}
\end{equation}
where a dot over a symbol indicates a time derivative, e.g., 
$\displaystyle \dot{\Phi}_i = \frac{\partial}{\partial t} \Phi_i$.
The terms on the r.h.s.\,of the first line of Eq.\,(\ref{eq:lagn1}) describe sections of the CPW between the SQUIDs. %
%
\begin{figure}
    %
    \includegraphics[width=\textwidth, keepaspectratio]{figures/system/circuit_diagram.png}
    \caption{Circuit diagram for a periodic lattice consisting of CPWs connected by symmetric SQUIDs at lattice sites $n$. 
    The dynamical fluxes $\Phi_i$ and $\Phi_{J,\,n}$ characterize the system.}
    \label{fig:circuit_diagram}
\end{figure}
%
%
\newpage
The second line of Eq.\,(\ref{eq:lagn1}) describes a periodic array of SQUIDs indexed by $n$ 
with effective capacitance $C_{J,\,n}$ and flux $\Phi_{J,\,n}$ at node $n$, where 
the subscript $J$ stands for the two Josephson junctions of a SQUID.
$\Phi_0 = h / (2 e)$ is the magnetic flux quantum.
The SQUIDs are separated by 
distance $\ell$ corresponding to the lattice constant of the one-dimensional (1D) SQUID array. 
The Josephson energy $E_{J,\,n}(t)$ of SQUID $n$ is tunable
by the externally applied magnetic flux $\Phi_{\text{ext},\,n}(t)$ through the SQUID 
(cp. Johansson, Eq.\,(6)): 
%
\begin{equation} \label{eq:squidenergy}
    E_{J,\,n}(t) = 2 \, \epsilon_{J,\,n} 
    \left\vert \cos\left(\pi \, \frac{\Phi_{\text{ext},\,n}(t)}{\Phi_0}\right)\right\vert \, \, ,
\end{equation}
%
where $\epsilon_{J,\,n}$ is a constant energy parameter.

We now make use of considerations for the SQUID parameters similar to those found in the treatment by Johansson, et. al. \cite{Johansson2010}.
We assume that the plasma frequency $\omega_p$ of the SQUIDs far exceeds other characteristic frequencies 
in the circuit so that oscillations in the phase across the SQUIDs have small amplitude, i.e., 
$\Phi_{J,\,n} / \Phi_0 \ll 1$,
and the SQUIDs are operated in the phase regime where $E_{J,\,n} \gg (2e)^2 / (2C_{J,\,n})$.
Using $\Phi_{J,\,n} / \Phi_0 \ll 1$ one may expand the cosine in Eq.\,(\ref{eq:lagn1}), 
resulting in a Lagrangian quadratic in $\Phi_{J,\,n}$ (dropping terms $E_{J,\,n}(t)$ 
independent of $\Phi_{J,\,n}$)
%
\begin{equation} \label{eq:lagn2}
\begin{split}
\mathcal{L} = \, & \frac{1}{2} \sum_i \left( \Delta x \, C_{0} \left(\dot{\Phi}_{i}\right)^{2} \, - \, 
\frac{\left(\Phi_{i+1}-\Phi_{i}\right)^{2}}{\Delta x \, L_{0}} \right)  \\[2mm]
& + \frac{1}{2} \sum_n \left( C_{J,\,n} \left(\dot{\Phi}_{J,\,n} \right)^{2} \, - \, 
 \left(\frac{2 \pi}{\Phi_0} \right)^2 E_{J,\,n}(t) \, \left( \Phi_{J,\,n} \right)^2 
\right) \, \, .
\end{split}
\end{equation}
%
Note that we make the identification $\Phi_i^{\,l} = \Phi_{J,\,n} = \Phi_i^{\,r}$ for the flux at SQUID $n$, 
where $\Phi_i^{\,l}$ and $\Phi_i^{\,r}$ are the fluxes at adjacent nodes to the left ($l$) and right ($r$) of 
node $n$, respectively (see Figure \ref{fig:circuit_diagram}). 

From now on, we assume that the intrinsic device parameters of all SQUIDs are equal, i.e., 
$C_{J,\,n} = C_J$, $\epsilon_{J,\,n} = \epsilon_J$ for all $n$ in Eqs.\,(\ref{eq:lagn2}), (\ref{eq:squidenergy}).  
Moreover, we assume that the SQUID energy $E_{J,\,n}(t)$ in Eq.\,(\ref{eq:squidenergy}) 
can be expanded in a static plus harmonic drive terms
%
\begin{equation} \label{eq:energyexp1}
E_{J,\,n}(t) = E_J^0 + \delta E_{J,\,n} \cos(\Omega \, t + \varphi_n) 
\end{equation}
%
where $\delta E_{J,\,n} < E_J^0$ and $\Omega$ is the frequency of the external drive 
$\Phi_{\text{ext}}(t)$ in Eq.\,(\ref{eq:squidenergy}).
%
As indicated in Eq.\,(\ref{eq:energyexp1}), we assume that the {\em static} part 
$E_J^0$ of $E_{J,\,n}(t)$ is the same for all SQUIDs (i.e., indepedent of $n$). 
This assumption is crucial for our 
treatment of the {\em static} system (realized by a static applied magnetic flux $\Phi_{\text{ext}}$
for all SQUIDs) as a 1D periodic lattice with period $\ell$. However, the 
time-dependent contribution in Eq.\,(\ref{eq:energyexp1})
may be modulated along the SQUID array, i.e., may differ for different SQUIDs $n$,
in terms of amplitudes $\delta E_{J,\,n}$ and phases $\varphi_n$.
This allows us to externally control the DCE radiation generated in the SQUID array
by varying the parameters $\delta E_{J,\,n}$ and $\varphi_n$.
However, we will assume for simplicity that the drive frequency 
is the same for all SQUIDs, i.e., $\Omega_n = \Omega$ for all $n$.
That is, we consider an amplitude modulation but no frequency modulation of the 
time-dependent contribution in Eq.\,(\ref{eq:energyexp1}). 


%%%%%%%%%%%%%%%%%%%%%%%%%%%%%%%%%%%%%%%%%%%%%%%%%%%%%%%%%%%%%%


\subsection{Quantization of the dynamic flux in the periodic SQUID array}
\noindent
To quantize the system, we first transform the Lagrangian $\mathcal{L}$ in Eq.\,(\ref{eq:lagn2})
into a Hamiltonian $\mathcal{H}$ by a Legendre transformation.
To this end, it is convenient to temporarily consider the fluxes $\Phi_i^{\,l}$, $\Phi_i^{\,r}$, and $\Phi_{J,\,n}$ 
at SQUIDs $n$ as independent dynamic variables (cp.\,remark below Eq.(\ref{eq:lagn2})) and define 
%
\begin{equation} \label{eq:ham1}
\mathcal{H} = \sum_{i} \frac{\partial\mathcal{L}}{\partial\dot{\Phi}_i} \, \dot{\Phi}_i 
\, + \, \sum_{n} \frac{\partial\mathcal{L}}{\partial\dot{\Phi}_{J,\,n}} \, \dot{\Phi}_{J,\,n}
\, - \, \mathcal{L} \, \, , 
\end{equation}
%
resulting in
%
\begin{equation} \label{eq:ham2}
\begin{split}
\mathcal{H} = \, & \frac{1}{2} \sum_i \left( \frac{P_i^{\,2}}{\Delta x \, C_{0}} \, + \, 
\frac{\left(\Phi_{i+1} - \Phi_{i}\right)^{2}}{\Delta x \, L_{0}} \right)  \\[2mm]
& + \frac{1}{2} \sum_n \left( \frac{\left(P_{J,\,n}\right)^2}{C_{J}} \, + \, 
 \left(\frac{2 \pi}{\Phi_0} \right)^2 E_{J,\,n}(t) \, \left( \Phi_{J,\,n} \right)^2 
\right)
\end{split}
\end{equation}
%
with the canonical conjugate momenta
%
\begin{subequations} \label{eq:mom}
\begin{eqnarray} 
P_i = \frac{\partial\mathcal{L}}{\partial\dot{\Phi}_i} = \Delta x \, C_0 \, \dot{\Phi}_i \label{eq:moma} \, \, \, , \\[2mm]
P_{J,\,n} = \frac{\partial\mathcal{L}}{\partial\dot{\Phi}_{J,\,n}} =  C_J \, \dot{\Phi}_{J,\,n} \, \, \,  . \label{eq:momb}
\end{eqnarray}
\end{subequations}
%
At this point, we identify again 
$\Phi_i^{\,l} = \Phi_{J,\,n} = \Phi_i^{\,r}$ and $P_i^{\,l} = P_{J,\,n} = P_i^{\,r}$
for the fluxes and conjugate momenta at SQUIDs $n$.
The system is quantized by turning the fluxes $\Phi_i(t)$ and conjugate momenta $P_i(t)$ 
to operators $\hat{\Phi}_i(t)$, $\hat{P}_i(t)$ with equal-time commutation relations
%
\begin{equation} \label{eq:cr} 
\left[\hat{\Phi}_i(t), \hat{P}_j(t) \right] = i \hbar \delta_{ij} \, \, , \quad 
\left[\hat{\Phi}_i(t), \hat{\Phi}_j(t) \right] = 0 \, \, , \quad 
\left[\hat{P}_i(t), \hat{P}_j(t) \right] = 0 \, \, \, , 
\end{equation}
%
with the identifications mentioned above; for example, 
$\left[\hat{\Phi}_i^{\,l}, \hat{P}_{J,\,n} \right] = 
\left[\hat{\Phi}_i^{\,r}, \hat{P}_{J,\,n} \right] = i \hbar$ (see Figure \ref{fig:circuit_diagram}).

%%%%%%%%%%%%%%%%%%%%%%%%%%%%%%%%%%%%%%%%%%%%%%%%%%%%%%

\subsection{Wave equation and boundary conditions}  \label{subsec:we}
\noindent
In the CPW between the SQUIDs, the Heisenberg equation of motion 
$\displaystyle \frac{d}{dt} \hat{P}_i = \frac{i}{\hbar} \left[\hat{\mathcal{H}}, \hat{P}_i \right]$ 
results in, using Eqs.\,(\ref{eq:ham2}) - (\ref{eq:cr}), 
%
\begin{equation} \label{eq:eom1}
C_0 L_0 \, \frac{d^2}{dt^2} \hat{\Phi}_i \, = \, 
\frac{1}{(\Delta x)^2} \left(\hat{\Phi}_{i+1} - 2 \hat{\Phi}_i +  \hat{\Phi}_{i-1} \right) \, \, .
\end{equation}
%
In the continuum limit $\Delta x \to 0$ writing $\hat{\Phi}_i(t) \to \hat{\Phi}(x,t)$ 
with a continuous position variable $x$ the r.h.s.\,of Eq.\,(\ref{eq:eom1}) becomes
$\displaystyle \frac{\partial^2\hat{\Phi}(x,t)}{\partial x^2}$. 
%\newpage
We thus obtain the wave equation for the dynamic flux operator
$\hat{\Phi}(x,t)$ in the CPW between the SQUIDs,
%
\begin{equation} \label{eq:eom2}
\frac{\partial^2\hat{\Phi}(x,t)}{\partial t^2} - v^2 \, \frac{\partial^2\hat{\Phi}(x,t)}{\partial x^2} = 0 \, \, ,
\end{equation}
%
where $v = 1 / \sqrt{C_0 L_0}$ is the wave propagation speed in the CPW.
We note that, in the continuum limit $\Delta x \to 0$, 
the momentum operator conjugate to $\hat{\Phi}(x,t)$ is given by,
instead of Eq.\,(\ref{eq:moma}), 
%
\begin{equation} \label{eq:mom2} 
\hat{\overline{P}}(x,t) := \frac{\hat{P}_{i}(t)}{\Delta x} = 
C_0 \, \frac{\partial}{\partial t} \hat{\Phi}(x,t)
\end{equation}
%
with equal-time commutation relations
%
\begin{equation} \label{eq:cr2} 
\left[\hat{\Phi}(x,t), \hat{\overline{P}}(x',t) \right] = i \hbar \delta(x - x') \, \, , \quad 
\left[\hat{\Phi}(x,t), \hat{\Phi}(x',t) \right] = 0 \, \, , \quad 
\left[\hat{\overline{P}}(x,t), \hat{\overline{P}}(x',t) \right] = 0 \, \, \, , 
\end{equation}
%
where $\delta(x)$ is the 1D delta function (with units of $1/x$).

Similarly, at the $n$th SQUID site, the Heisenberg equations of motion
$\displaystyle \frac{d}{dt} \hat{P}_{J,\,n} = \frac{i}{\hbar} \left[\hat{\mathcal{H}}, \hat{P}_{J,\,n} \right]$ 
results in, using Eqs.\,(\ref{eq:ham2}) - (\ref{eq:cr}) with the identifications noted below
Eq.\,(\ref{eq:momb}), 
%
\begin{equation}\label{eq:BC_discrete}
C_{J} \, \frac{d^2}{dt^2} \hat{\Phi}_{J,\,n} \, + \left(\frac{2 \pi}{\Phi_{0}} \right)^{2} E_{J,\,n}(t) \, \hat{\Phi}_{J, \, n}
\, + \frac{1}{L_{0}} 
\left( \frac{\hat{\Phi}_{J,\,n} - \hat{\Phi}_{i-1}}{\Delta x} - \frac{\hat{\Phi}_{i+1} - \hat{\Phi}_{J,\,n}}{\Delta x} \right)
\, = \, 0 \, \, ,
\end{equation}
%
where $\hat{\Phi}_{i-1}$ and $\hat{\Phi}_{i+1}$ are adjacent nodes to the left and right of 
$\hat{\Phi}_i^{\,l} = \hat{\Phi}_{J,\,n} = \hat{\Phi}_i^{\,r}$, respectively (see Figure \ref{fig:circuit_diagram}). 
In the continuum limit $\Delta x \to 0$ we obtain the boundary condition for $\hat{\Phi}(x, t)$ 
at the SQUID sites $x_n$
%
\begin{equation}\label{eq:BC_field_orig}
C_{J} \, \frac{\partial^2 \hat{\Phi}(x_n, t)}{\partial t^2} \, + 
\left(\frac{2 \pi}{\Phi_{0}}\right)^{2} E_{J,\,n}(t) \, \hat{\Phi}(x_n, t) \, + 
\frac{1}{L_{0}}\left(\left.\frac{\partial \hat{\Phi}(x, t)}{\partial x}\right|_{x_n^{-}}
- \, \left.\frac{\partial \hat{\Phi}(x,t)}{\partial x}\right|_{x_n^{+}}\right) \, = \, 0 \, \, ,
\end{equation}
%
where $\displaystyle \left.\frac{\partial \hat{\Phi}}{\partial x}\right|_{x_n^{-}}$
denotes a derivative from the left ($-$) of $x_n$ evaluated at $x_n$ and
$\displaystyle \left.\frac{\partial \hat{\Phi}}{\partial x}\right|_{x_n^{+}}$
denotes a derivative from the right ($+$) of $x_n$ evaluated at $x_n$. 

The SQUID energy $E_{J,\,n}(t)$ is given by Eq.\,(\ref{eq:energyexp1}). 
%
% \newpage
In the absence of the SQUIDs, i.e., $C_J = E_{J,\,n}(t) = 0$, 
Eq.\,(\ref{eq:BC_field_orig}) shows that $\partial \hat{\Phi} / \partial x$ is 
continuous everywhere, and the solution of the wave equation (\ref{eq:eom2})
are simple harmonic waves $\hat{\Phi}(x,t) \sim \exp\left[i (q x - \omega \, t) \right]$
with angular frequency $\omega$, 1D wave vector $q$, and dispersion relation 
$\omega(q) = v |q|$
%
\footnote{We denote the 1D wave vector for the free CPW by the symbol $q$ to distinguish it 
from the wave vector $k$ of the Bloch waves obtained for the periodic SQUID array in the static case
obtained in Section \ref{subsec:kpsol}. \label{footnote:q}}.
However, in the presence of the SQUIDs the boundary condition  
(\ref{eq:BC_field_orig}) introduces discontinuities (jumps) of $\partial \hat{\Phi} / \partial x$ 
at the SQUID sites $x_n = \ell n$ and $\hat{\Phi}(x,t)$ are no longer simple harmonic waves (cp.\,Section \ref{sec:kp} below).

%%%%%%%%%%%%%%%%%%%%%%%%%%%%%%%%%%%%%%%%%%%%%%%%%%%%%%%%%%%%%%%%%%%%%%%%%%%%%%%%%%%%%%%%%%%%%%%%%%%%%%
%%%%%%%%%%%%%%%%%%%%%%%%%%%%%%%%%%%%%%%%%%%%%%%%%%%%%%%%%%%%%%%%%%%%%%%%%%%%%%%%%%%%%%%%%%%%%%%%%%%%%%

\section{Static case: Relation to the Kronig-Penney model} 
\label{sec:kp}

\noindent
In this section we consider the static case, in which $\delta E_{J,\,n} = 0$ in Eq.\,(\ref{eq:energyexp1})
and the SQUID energy $E_{J,\,n} = E_J^0$ is constant. In this case, the boundary condition 
(\ref{eq:BC_field_orig}) is given by
%
\begin{equation}\label{eq:BC_field_static_orig}
C_{J} \, \frac{\partial^2 \hat{\Phi}(x_n, t)}{\partial t^2} \, + 
\left(\frac{2 \pi}{\Phi_{0}}\right)^{2} E_J^0 \, \hat{\Phi}(x_n, t) \, + 
\frac{1}{L_{0}}\left(\left.\frac{\partial \hat{\Phi}(x, t)}{\partial x}\right|_{x_n^{-}}
- \, \left.\frac{\partial \hat{\Phi}(x,t)}{\partial x}\right|_{x_n^{+}}\right) \, = \, 0
\end{equation}

%%%%%%%%%%%%%%%%%%%%%%%%%%%%%%%%%%%%%%%%%%%%%%%%%%%%%%%%%%%%%%%%%%%%%%%%%%%%%%%%%%%%%%%%%%%%%%%%%%%%%

\subsection{Boundary condition in the frequency domain}\label{sec:BC_in_frequency_domain}
%
\noindent
We will generally work in the frequency domain instead of the time domain by 
expanding $\hat{\Phi}(x, t)$ in harmonic modes 
$\hat{\Phi}_{\omega}(x, t) = \hat{\varphi}(x,\omega) \exp(- i \omega \, t)$ of frequency $\omega$
(see Eqs.\,(\ref{eq:flux_field2}) and (\ref{eq:flux_field_omega}) below).
In the frequency domain the boundary condition for the static case in Eq.\,(\ref{eq:BC_field_static_orig}) 
translates to a boundary condition for $\hat{\varphi}(x,\omega)$:
%
\begin{equation}\label{eq:BC_freq_static}
- C_{J} \, \omega^2 \hat{\varphi}(x_n, \omega) \, + 
\left(\frac{2 \pi}{\Phi_{0}}\right)^{2} E_J^0 \, \hat{\varphi}(x_n, \omega) \, + 
\frac{1}{L_{0}}\left(\left.\frac{\partial \hat{\varphi}(x, \omega)}{\partial x}\right|_{x_n^{-}}
- \, \left.\frac{\partial \hat{\varphi}(x,\omega)}{\partial x}\right|_{x_n^{+}}\right) \, = \, 0
\end{equation}
%
Note that Eq.\,(\ref{eq:BC_freq_static}) only holds in the static case because 
for time-dependent SQUID energy $E_{J,\,n}(t)$ the term $E_{J,\,n}(t) \, \hat{\Phi}(x_n, t)$ 
in Eq.\,(\ref{eq:BC_field_orig}) generates a coupling of modes with different frequencies $\omega$.
%
%The physical unit of the field operator $\hat{\Phi}(x,t)$ in Eq.\,(\ref{eq:BC_field_orig}) is determined 
%by the first commutation relation in Eq.\,(\ref{eq:cr2}) with $\hat{\overline{P}}(x,t)$ from Eq.\,(\ref{eq:mom2}).
%This implies that the physical unit of the field operator is determined by  
%%
%\begin{equation} \label{eq:phiunit} 
%\left[ \hat{\Phi}^2 \right] = \left[ \frac{\hbar \cdot \text{s}}{C_0 \cdot \text{m}} \right] \, \, .
%\end{equation} 
%%
In large parts of this thesis we will use unitless variables by expressing lengths in units of $\ell$
and times in units of $\ell/v$ where $\ell$ is the distance between SQUIDs in the periodic array 
(see Figure \ref{fig:circuit_diagram})
and $v = 1 / \sqrt{C_0 L_0}$ is the speed of light in the CPW in Eq.\,(\ref{eq:eom2}).  
%
%We thus can introduce a unitless field operator by
%%
%\begin{equation} \label{eq:ufo2}
%\hat{\phi}(x,t) := \sqrt{\frac{2 C_0 v}{\hbar}} \, \hat{\Phi}(x,t) \, \, ,
%\end{equation}
%
%where here $x$, $t$ are the unitless versions of position and time and 
%the factor of 2 in the square root is for convenience.
The boundary condition (\ref{eq:BC_freq_static}) can be expressed in terms of unitless parameters
by multiplying both sides of Eq.\,(\ref{eq:BC_freq_static}) by $\ell \, L_0$ resulting in
%
\footnote{We use the same symbol $x$ for the original position variable and its 
unitless version (position in units of $\ell$) to keep the notation simple.}
%
\begin{equation}\label{eq:BC_freq_static_unitless}
\epsilon(\omega) \hat{\varphi}(n, \omega) \, = \, 
\left.\frac{\partial \hat{\varphi}(x, \omega)}{\partial x}\right|_{n^{+}}
- \, \left.\frac{\partial \hat{\varphi}(x,\omega)}{\partial x}\right|_{n^{-}}
\end{equation}
%
where $n$ is the position of the $n$th SQUID in units of $\ell$ and 
%
\begin{equation} \label{eq:epsilon}
\epsilon(\omega) \, = \, \ell \, L_0 \left[ - C_{J} \, \omega^2 + 
\left(\frac{2 \pi}{\Phi_{0}}\right)^{2} E_J^0 \right] 
\end{equation}
%
is a unitless parameter.

From now on we assume for simplicity that the SQUID plasma frequency $\omega_p$ is sufficiently 
large so that we can neglect the term $C_{J} \, \omega^2$ in Eq.\,(\ref{eq:epsilon}) compared to 
the other term \cite{Johansson2010}
(cp.\,text below Eq.\,(\ref{eq:squidenergy})), i.e., we will use the $\omega$-independent parameter
\newpage
%
\begin{equation} \label{eq:epsilon2}
\epsilon \, = \, \ell \left(\frac{2 \pi}{\Phi_{0}}\right)^{2} E_J^0 L_0 \, \, .
\end{equation}
%
A physical interpretation of the unitless parameter $\epsilon$ can be obtained by writing
%
\begin{equation} \label{eq:epsilon_ip}
\epsilon \, = \, \frac{\ell}{L_{\text{eff}}^0} \, \, \, \, \text{with} 
\, \, \, \, L_{\text{eff}}^0 = \left(\frac{\Phi_0}{2 \pi}\right)^{2} \frac{1}{E_J^0 L_0} \, \, .
\end{equation}
%
For a single SQUID terminating a CPW, the parameter $L_{\text{eff}}^0$ can be interpreted
as an effective length that gives the distance from the SQUID to a perfectly reflecting mirror 
\cite{Johansson2010}. Thus, $\epsilon$ corresponds to the ratio of the lattice constant $\ell$
of the SQUID array to this effective length $L_{\text{eff}}^0$. 


%%%%%%%%%%%%%%%%%%%%%%%%%%%%%%%%%%%%%%%%%%%%%%%%%%%%%%%%%%%%%%

\subsection{Kronig-Penney model}
\label{subsec:kp}
\noindent
The Kronig-Penney model is a simplified model for an electron in a 1D periodic potential
(see \cite{Kittel1996_Solid_State}).
Modeling the periodic potential as a Dirac comb (see Figure \ref{fig:dirac_comb}) with amplitude $u>0$
and lattice constant $\ell$, the time-independent Schr\"odinger equation reads
%
\begin{equation} \label{eq:kpschrod}
- \frac{\hbar^2}{2 m} \frac{d^2}{dx^2} \psi(x) + u \, \ell \sum_n \delta(x - n \ell) \psi(x) = E \psi(x) \, \, , 
\end{equation}
%
where $\psi(x)$ is the wave function, $m$ the mass, and $E$ the energy 
of the electron.
%
\begin{figure}
    %
    \includegraphics[width=\textwidth, keepaspectratio]{figures/system/dirac_comb.png}
    \caption{Diagrammatic representation of Dirac comb potential as an infinite series of Dirac delta distributions placed at intervals $\ell$.}
    \label{fig:dirac_comb}
\end{figure}
%
This equation can be made unitless by multiplying both sides
with $\displaystyle \frac{2 m \, \ell^2}{\hbar^2}$ which yields
%
\begin{equation} \label{eq:kpschrod_unitless}
- \psi''(x) + \epsilon \sum_n \delta(x - n) \psi(x) = {\cal E} \psi(x) \, \, , \quad \text{(Kronig-Penney, unitless)} 
\end{equation}
%
where the position $x$ is expressed in units of $\ell$, the prime symbol denotes a derivative
with respect to $x$, and
%
\begin{equation} \label{eq:kpschrod_param}
\epsilon = \frac{2 m u \, \ell^2}{\hbar^2} \, \, , \quad {\cal E} = \frac{2 m E \ell^2}{\hbar^2}
\end{equation}
%
are unitless (re-scaled) versions of $u$ and $E$.
\newpage
Integrating Eq.\,(\ref{eq:kpschrod_unitless}) over a small region about  
$n$ according to $\displaystyle \int_{n - \delta x}^{n + \delta x} dx$ \,
followed by the limit $\delta x \to 0$ yields
a boundary condition at $n$ in terms of a jump of the first derivative $\psi'(x)$ (see \cite{Kittel1996_Solid_State}).
%
\begin{equation}\label{eq:kpjump}
\epsilon \psi(n,{\cal E}) \, = \, 
\left. \psi'(n,{\cal E}) \right|_{n^{+}} - \left. \psi'(n,{\cal E}) \right|_{n^{-}} \, \, .
\quad \text{(Kronig-Penney, unitless)}
\end{equation}

Comparison with Eq.\,(\ref{eq:BC_freq_static_unitless}) shows that the form of the boundary
conditions for the field operator $\hat{\varphi}(x, \omega)$ in the 1D periodic SQUID array
(static case)
and the electron wave function $\psi(x,{\cal E})$ in the Kronig-Penney model 
are identical. However,
in the absence of the periodic potential in the Kronig-Penney model, i.e., $u=\epsilon=0$, 
the solution of Eq.\,(\ref{eq:kpschrod_unitless}) 
are harmonic waves $\psi(x,{\cal E}) \sim \exp(i q x)$ with 1D wave vector $q$ and dispersion relation 
${\cal E}(q) = q^2$ (corresponding to 
$\displaystyle E(q) = \hbar^2 q^2 / (2 m)$ in original variables).
In contrast, for the free CPW we found the dispersion relation
$\omega(q) = |q|$ in unitless variables (corresponding to $\omega(q) = v |q|$ in original variables,
see text below Eq.\,(\ref{eq:BC_field_orig}) and footnote \ref{footnote:q} on page \pageref{footnote:q}).
That is, $\omega(q)$ in the free CPW is a linear 
function of $q$ instead of quadratic, which is a consequence of the fact that the particles 
in the CPW are massless photons whereas the electrons in the Schr\"odinger equation (\ref{eq:kpschrod})
have a finite mass $m$. 
%
However, except for this difference of the dispersion relation, the solution for the 
field operator $\hat{\varphi}(x, \omega)$ in the 1D periodic SQUID array 
with boundary condition (\ref{eq:BC_freq_static_unitless}) at the SQUID sites $n$ for the static case
is analogous to that of the Kronig-Penney model. 

%%%%%%%%%%%%%%%%%%%%%%%%%%%%%%%%%%%%%%%%%%%%%%%%%%%%%%%%%%%%%

\subsection{Solution of the Kronig-Penney-type model for the 1D SQUID array}
\label{subsec:kpsol}

\noindent
This subsection uses unitless variables
by expressing lengths in units of $\ell$ and times in units of $\ell/v$ as discussed above
\footnote{We use the same symbols $x$, $q$, $k$, $\omega$, etc.\,for the original variables 
and their unitless versions to keep the notation simple. It will be clear from 
the context and/or by explicit marking which version of variables are used.}.
%
In the frequency domain, the wave equation for the CPW between the SQUIDs is given by
Eq.\,(\ref{eq:eom2}) with 
$\Phi_{\omega}(x, t) = \varphi(x,\omega) \exp(- i \omega \, t)$ 
(see Section \ref{sec:BC_in_frequency_domain}), which results in
\footnote{In this subsection we skip the hat symbol for operators since the operator property 
of $\varphi(x,\omega)$ is not used.}
%
\begin{equation} \label{eq:eom3}
\varphi''(x,\omega) + \omega^2 \varphi(x,\omega) = 0 \, \, .
\end{equation}
% 
The boundary conditions at the SQUID sites $n$, in the static case, 
are given by Eq.\,(\ref{eq:BC_freq_static_unitless}):
%
\begin{equation}\label{eq:BC_freq_static_unitless_2}
\epsilon \varphi(n, \omega) \, = \, 
\left.\frac{\partial \varphi(x, \omega)}{\partial x}\right|_{n^{+}}
- \, \left.\frac{\partial \varphi(x,\omega)}{\partial x}\right|_{n^{-}} \quad \text{for all integer} \, \, \, n \, \, \, .
\end{equation}
%
Note that, by using unitless variables, the only system-specific parameter in 
Eqs.\,(\ref{eq:eom3}), (\ref{eq:BC_freq_static_unitless_2}) is the unitless energy parameter 
$\epsilon$ in Eq.\,(\ref{eq:epsilon2}). 

The strategy for solving Eqs.\,(\ref{eq:eom3}) and (\ref{eq:BC_freq_static_unitless_2}) 
is to use an ansatz for $\varphi_n(x)$ for each section $n$ of the CPW
between SQUIDs as a linear combination of right-moving
and left-moving plane waves with 1D wave vector $q>0$ as in a free CPW:
%
\begin{equation} \label{eq:kpansatz}
\varphi_n(x) \, = \, A_n \, \exp\left[i q (x-n) \right] \, + \, 
B_n \, \exp\left[-i q (x-n) \right] \, \, , \quad n-1 < x < n \, \, ,
\end{equation}
%
and to determine the $n$-dependent coefficients $A_n$, $B_n$ by the boundary conditions 
(\ref{eq:BC_freq_static_unitless_2}) using a transfer matrix approach.
The extra terms $\exp(- i q n)$, $\exp(i q n)$ in Eq.\,(\ref{eq:kpansatz}) were introduced 
for computational convenience (see Mathematica notebook in the Appendix).
%
We find solutions for the Kronig-Penney-type model defined by 
Eqs.\,(\ref{eq:eom3}), (\ref{eq:BC_freq_static_unitless_2})
in terms of Bloch functions
%
\begin{equation} \label{eq:psi1}
\psi_{\nu,\,k}(x) = e^{i k x} u_{\nu,\,k}(x) \, \, ,   
\end{equation}
%
where $k$ is a 1D Bloch wave vector, which depends in a nontrivial way on the wave vector $q$ used in the ansatz (\ref{eq:kpansatz}), and $\nu$ is the band index \cite{Ashcroft1976}.
Since $\omega = q$ in unitless variables (corresponding to $\omega(q) = v q$ in original variables,
and using $q>0$ in Eq.\,(\ref{eq:kpansatz})),
the dependence $k(q)$ directly translates to a dependence $k(\omega)$, corresponding to the 
inverse of the dispersion relation $\omega_{\,\nu}(k)$ in frequency band $\nu$ for the Kronig-Penney-type model (see Figure \ref{fig:komega}).

The function  $u_{\nu,\,k}(x)$ in Eq.\,(\ref{eq:psi1}) has period $\ell$, i.e., 
period $1$ in unitless variables, which means that
$u_{\nu,\,k}(x + n) = u_{\nu,\,k}(x)$ for all integers $n$. 
Thus, $u_{\nu,\,k}(x)$ is completely specified by the domain $x \in [0,1]$ 
and periodic extension beyond this domain.
The Bloch functions $\psi_{\nu,\,k}(x)$ in Eq.\,(\ref{eq:psi1})
are unitless, orthogonal with respect to the 1D wave vector $k$, and normalized such that
%
\begin{equation} \label{eq:psi1_norm_orig}
\int_{-\infty}^{\infty} dx \, \psi^*_{\nu,\,k}(x) \psi_{\nu',\,k'}(x) = 2 \pi \, \delta_{\nu \nu'} \, \delta(k - k') \, \, .
\end{equation}
%
Examples of the functions $u_{\nu,\,k}(x)$ and $\psi_{\nu,\,k}(x)$
are shown in Figs.\,\ref{fig:u} and \ref{fig:psi}, respectively. 

The introduction of a periodic potential leads to a Bloch band structure 
with frequency bands $\nu$ similar to electrons in crystals
\cite{Ashcroft1976} or photons in photonic crystals \cite{Joannopoulos2008}.
The frequency bands are determined by the relation
%
\begin{equation}\label{eq:disp_rel}
    \cos{k} = \cos{\omega} + \frac{\epsilon}{2\omega}\sin{\omega}
\end{equation}
%
with $\epsilon$ defined in Eq.\,(\ref{eq:epsilon2}) and included in the boundary condition 
(\ref{eq:BC_freq_static_unitless_2}). 
Since the left-hand side of Eq.\,(\ref{eq:disp_rel}) has range $[-1,1]$ allowed frequencies 
$\omega$ are determined by the condition (see Figure\,\ref{fig:band_condition})
%
\begin{equation}\label{eq:bands_condition}
    \left|\cos{\omega} + \frac{\epsilon}{2\omega}\sin{\omega}\right| \leq 1 \, \, .
\end{equation}
%
For allowed frequencies $\omega$, the function $k(\omega)$ is given by
(using the principal value range of the $\arccos$ function) (see Figure\,\ref{fig:komega})
%
\begin{equation}\label{eq:disp_rel_k}
    k(\omega) = \arccos\left[ \cos{\omega} + \frac{\epsilon}{2\omega}\sin{\omega} \right] \, \in \, [0, \pi] \, \, .
\end{equation}
%
Conversely, the dispersion relation of the Kronig-Penney-type model is given by solutions 
$\omega_{\, \nu}(k)$ of Eq.\,(\ref{eq:disp_rel_k}), where the 
band index $\nu$ labels different solutions $\omega_{\, \nu}$ for given $k \in [-\pi, \pi]$
in the first Brillouin zone of the reciprocal lattice (in unitless variables where the lattice
constant is 1).
Since $\omega_{\, \nu}(k) = \omega_{\, \nu}(-k)$ by symmetry of the lattice, 
it is sufficient to consider positive wave vectors $k>0$. 
A solution 
of Eq.\,(\ref{eq:disp_rel_k}) for $\omega_{\, \nu}(k)$ obtained 
by graphically inverting the function $k(\omega)$ using the 
Mathematica function ParametricPlot (see Appendix) 
is shown in Figure \ref{fig:omegak}
%
\footnote{The dispersion relation $\omega_{\, \nu}(k)$ resulting from Eq.\,(\ref{eq:disp_rel_k}) cannot be 
calculated in closed form. Our calculations will therefore use the inverse function $k(\omega)$,
which is directly given in analytical form by Eq.\,(\ref{eq:disp_rel_k}). See Section \ref{subsec:sq}.}.

\begin{figure}
    %
    \includegraphics[width=0.8\textwidth, keepaspectratio]{figures/system/u.png}
    \caption{(a) Real part and (b) imaginary part of the function $u_{\nu,\,k}(x)$ 
    for $0 \le x \le 1$ for the Bloch function 
    $\psi_{\nu,\,k}(x) = e^{i k x} u_{\nu,\,k}(x)$ in Eq.\,(\ref{eq:psi1}) for $\epsilon=5$. 
    Here $\omega=2$, i.e., $k = k(2)$ and $\nu=1$ (see Figure\,\ref{fig:omegak} below).}
    \label{fig:u}
\end{figure}

\begin{figure}
    %
    \includegraphics[width=0.8\textwidth, keepaspectratio]{figures/system/psi.png}
    \caption{(a) Real part and (b) imaginary part of the Bloch function 
    $\psi_{\nu,\,k}(x) = e^{i k x} u_{\nu,\,k}(x)$ in Eq.\,(\ref{eq:psi1}). 
    Parameter values as in Figure\,\ref{fig:u}.}  
    \label{fig:psi}
\end{figure}
%
\begin{figure}
    %
    \includegraphics[width=0.7\textwidth, keepaspectratio]{figures/system/cosk.png}
    \caption{Frequency bands and gaps for the Kronig-Penney-type model defined by 
    Eqs.\,(\ref{eq:eom3}) and (\ref{eq:BC_freq_static_unitless_2}).
    Shown is the function
    $f(\omega) = \cos{\omega} + \frac{\epsilon}{2 \omega} \sin{\omega}$ for $\epsilon=5$. 
    The relation $f(\omega) = \cos(k)$ can be satisfied for real $k$ if and only if 
    $|f(\omega)| \le 1$ (see Eq.\,(\ref{eq:bands_condition})). 
    Modes with $|f(\omega)| \le 1$
    are allowed and can freely propagate through the one-dimensional periodic structure
    (see Figure\,\ref{fig:dirac_comb}).
    Modes with $|f(\omega)| > 1$ decay exponentially and are therefore forbidden, 
    resulting in frequency gaps (shown shaded in the figure).}
    \label{fig:band_condition}
\end{figure}
%
\begin{figure}
    %
    \includegraphics[width=0.7\textwidth, keepaspectratio]{figures/system/komega.png}
    \caption{Inverse dispersion relation $k(\omega) \in [0,\pi]$ 
    for allowed frequencies $\omega$ given by Eq.\,(\ref{eq:disp_rel_k}) for $\epsilon=5$. 
    The band gaps are shown shaded.}  
    \label{fig:komega}
\end{figure}

\begin{figure}
    %
    \includegraphics[width=0.4\textwidth, keepaspectratio]{figures/system/omegak.png}
    \caption{Dispersion relation $\omega(k)$ with frequency bands $\nu$ resulting from 
    $\cos k = \cos{\omega} + \frac{\epsilon}{2 \omega} \sin{\omega}$ for $\epsilon=5$
    in the reduced zone scheme (see Eq.\,(\ref{eq:disp_rel_k})). 
    The frequency gaps are shown shaded. The $k$-interval $[-\pi,\pi]$ 
    is the first Brillouin zone of the reciprocal lattice in 
    unitless variables where the lattice constant is equal to 1.
    Modes with $k>0$ are moving to the right and modes with $k<0$
    are moving to the left.}
    \label{fig:omegak}
\end{figure}

%%%%%%%%%%%%%%%%%%%%%%%%%%%%%%%%%%%%%%%%%%%%%%%%%%%%%%%%%%%%%%%%%%%%%%%%%%%%%%%%%%%%%%%%%%%%%%%%%%%%%%%%%%%%%%
%%%%%%%%%%%%%%%%%%%%%%%%%%%%%%%%%%%%%%%%%%%%%%%%%%%%%%%%%%%%%%%%%%%%%%%%%%%%%%%%%%%%%%%%%%%%%%%%%%%%%%%%%%%%%%

\section{Harmonic drive} \label{sec:hd}

\noindent
The goal of this section is to find solutions for the dynamic flux field $\hat{\Phi}(x,t)$ 
introduced in Section \ref{subsec:we} in the presence of a time-dependent, harmonic applied 
magnetic flux with frequency $\Omega$ resulting in the time-dependent SQUID energy $E_{J,\,n}(t)$ 
in Eq.\,(\ref{eq:energyexp1}). In terms of unitless variables, $E_{J,\,n}(t)$ is given by
%
\begin{equation} \label{eq:energyexp_alt}
\begin{split}
\epsilon_n(t) \, & = \, \epsilon \, + \, \delta \epsilon_n^0 \, \cos(\Omega \, t + \varphi_n) \\[2mm]
& =: \, \epsilon \, + \, \delta \epsilon_n(t) 
\end{split}
\end{equation}
%
with $\epsilon = \ell \, \displaystyle{\left(\frac{2 \pi}{\Phi_{0}}\right)^{2}} E_J^0 L_0\,$
from Eq.\,(\ref{eq:epsilon2}) and the unitless amplitude 
%
\begin{equation} \label{eq:deltaepsilon_alt}
    \delta \epsilon_n^0 \, := \, \ell \left(\frac{2 \pi}{\Phi_{0}}\right)^{2} \,
\delta E_{J,\,n} \, L_0 \, \, .
\end{equation}
%
To proceed, it will be convenient to represent the flux field $\hat{\Phi}(x,t)$
in second-quantized form.

%%%%%%%%%%%%%%%%%%%%%%%%%%%%%%%%%%%%%%%%%%%%%%%%%%%%%%%%%%%%%%%%%%%%%%%%%%%%%%%%%%%%%%%

\newpage

\subsection{Dynamic flux field in second-quantized form}
\label{subsec:sq}

\noindent
We start this Section using original variables for clarity, and later turn to unitless variables to 
simplify the notation and facilitate the numerical implementation. 
For each section $n$ of the CPW between SQUIDs $n-1$ and $n$ we expand the field operator $\hat{\Phi}(x,t)$
introduced below Eq.\,(\ref{eq:eom1}), with time-dependent 
boundary conditions (\ref{eq:BC_field_orig}) at the SQUID sites 
$x_n = n \ell$, in second quantized form using annihilation and creation operators $\hat{a}_n(\nu,k)$ and 
${\hat a}_n^\dagger(\nu,k)$ for wave vector $k$ and frequency band $\nu$
(see Figure \,\ref{fig:system} on page \pageref{fig:system} for a depiction of 
sites $n$ and sections $n$ of the SQUID array):  
%
\begin{equation} \label{eq:flux_field_orig}
\begin{split}
    \hat{\Phi}_n(x,t) \, = \, \sqrt{\frac{\hbar}{2 C_0}} \, \sum_{\nu=1}^{\infty} \, 
    \int\limits_{-\pi/\ell}^{\pi/\ell}\frac{dk}{2 \pi} \frac{1}{\sqrt{\omega_{\,\nu}(k)}}
    \left[ \, \hat{a}_n(\nu,k) \psi_{\nu,\,k}(x) \, e^{-i \omega_{\,\nu}(k) \, t} \, + \, 
    \hat{a}_n^{\dagger}(\nu,k) \psi_{\nu,\,k}^*(x) \, e^{i \omega_{\,\nu}(k) \, t} \, \right] \, \, , \\
    \quad x_{n-1} < x < x_n \, \, ,
\end{split}
\end{equation}
%
where $\psi_{\nu,\,k}(x)$ are Bloch functions with dispersion relation $\omega_{\,\nu}(k)$ defined for 
the {\em static} system, 
and the $k$-integration is over the first Brillouin zone 
$\displaystyle{\left[-\frac{\pi}{\ell}\, , \, \frac{\pi}{\ell} \right]}$
of the reciprocal lattice (see Section \ref{subsec:kpsol}). The Bloch functions have the form (see Eq.\,(\ref{eq:psi1})) 
%
\begin{equation} \label{eq:psi2}
\psi_{\nu,\,k}(x) = e^{i k x} u_{\nu,\,k}(x) \, \, ,
\end{equation}
%
where $u_{\nu,\,k}(x)$ has period $\ell$ so that $u_{\nu,\,k}(x + n \ell) = u_{\nu,\,k}(x)$
for all integers $n$.
The Bloch functions $\psi_{\nu,\,k}(x)$ are unitless, orthogonal with respect to 
the 1D wave vector $k$, normalized such that
%
\begin{equation} \label{eq:psi1_norm_orig2}
\int_{-\infty}^{\infty} dx \, \psi^*_{\nu,\,k}(x) \psi_{\nu',\,k'}(x) = 2 \pi \, \delta_{\nu \nu'} \, \delta(k - k') \, \, ,
\end{equation}
%
and fulfill the completeness relation
%
\begin{equation} \label{eq:psi1_comp_orig}
\sum_{\nu=1}^{\infty} \, 
\int\limits_{-\pi/\ell}^{\pi/\ell}\frac{dk}{2 \pi} \, 
\psi_{\nu,\,k}(x) \psi^*_{\nu,\,k}(x') = \delta(x-x') \, \, .
\end{equation}
%
The index $n$ in Eq.\,(\ref{eq:flux_field_orig}) indicates
that $\hat{\Phi}_n(x,t)$ is defined in the domain
$x_{n-1} < x < x_n$ corresponding to section $n$ of the CPW. Field operators $\hat{\Phi}_n(x,t)$,
$\hat{\Phi}_{n+1}(x,t)$ in subsequent sections $n$, $n+1$ are coupled to each other
by the time-dependent 
boundary condition (\ref{eq:BC_field_orig}) at SQUID site $x_n$.
The operators $\hat{a}_n(\nu,k)$, $\hat{a}_n^{\dagger}(\nu,k)$ in Eq.\,(\ref{eq:flux_field_orig})
can be interpreted as expansion coefficients of 
$\hat{\Phi}_n(x,t)$ in Bloch modes $\psi_{\nu,\,k}(x) \exp\left[-i \omega_{\,\nu}(k) \, t \right]$ in section $n$.
Conversely, the Bloch functions $\psi_{\nu,\,k}(x)$ are defined for the global system by construction 
and do not depend on $n$.

For each section $n$ of the CPW the annihilation and creation operators in 
Eq.\,(\ref{eq:flux_field_orig}) obey the commutation relations
%
\begin{equation} \label{eq:cra_orig}
\begin{split}
    & \left[ \hat{a}_n(\nu,k),{\hat a}_n^\dagger(\nu',k') \right] \, = \, 
    2 \pi \delta_{\nu \nu'} \, \delta(k - k') \, \, , \\[2mm]
    & \left[ \hat{a}_n(\nu,k),{\hat a}_n(\nu',k') \right] \, = \, 
    \left[ \hat{a}_n^\dagger(\nu,k),{\hat a}_n^\dagger(\nu',k') \right] \, = \, 0 \, \, ,
\end{split}
\end{equation}
%
and have units of $\text{length}^{1/2}$.
The prefactor $\displaystyle{\sqrt{\frac{\hbar}{2 C_0}}}$ in Eq.\,(\ref{eq:flux_field_orig}) 
is determined by the equal-time commutation relation for section $n$ of the CPW,
$\left[\hat{\Phi}_n(x,t), \hat{\overline{P}}_n(x',t) \right] = i \hbar \delta(x - x')$ 
in Eq.\,(\ref{eq:cr2}), using Eqs.\,(\ref{eq:cra_orig}) and (\ref{eq:psi1_comp_orig}).

The dispersion relation $\omega_{\,\nu}(k)$ used in
Eq.\,(\ref{eq:flux_field_orig}) cannot be calculated in closed form. 
Conversely, the inverse function $k(\omega) \in [0,\pi]$ is uniquely defined for given 
$\omega = \omega_{\,\nu}$ and is directly available in analytical form by Eq.\,(\ref{eq:disp_rel_k}) (see Figure \ref{fig:komega})
%
\footnote{In what follows, $k(\omega) \in [0,\pi]$ denotes the {\em positive} solution for $k$ of 
Eq.\,(\ref{eq:disp_rel}) in the reduced zone scheme given by Eq.\,(\ref{eq:disp_rel_k})
and Figure\,\ref{fig:komega}.\label{foot:k}}.
%
It is therefore convenient to perform a substitution of integration variables $k \to \omega$
in Eq.\,(\ref{eq:flux_field_orig}), which gives
%
\begin{equation} \label{eq:flux_field_orig_omega1}
\begin{split}
    \hat{\Phi}_n(x,t) \, = \, \sqrt{\frac{\hbar}{2 C_0}} \, &
    \int_{0, \, \text{w/o gaps}}^{\infty} \, \frac{d\omega}{2 \pi} \, \left| \frac{d k}{d \omega} \right|
        \frac{1}{\sqrt{\omega}} \\[2mm]
    & \times \left[ \, \hat{a}_n\left[ \nu, k(\omega) \right] \psi_{\nu, \, k(\omega)}(x) \, e^{-i \omega \, t} \, + \,
    \hat{a}_n^{\dagger}\left[ \nu, k(\omega) \right] \psi_{\nu, \, k(\omega)}^*(x) \, e^{i \omega \, t} \right. \\[2mm]
    & \quad \left. + \, \hat{a}_n\left[ \nu, -k(\omega) \right] \psi_{\nu, \, -k(\omega)}(x) \, e^{-i \omega \, t} \, + \,
    \hat{a}_n^{\dagger}\left[ \nu, -k(\omega) \right] \psi_{\nu, \, -k(\omega)}^*(x) \, e^{i \omega \, t} \, \right] \, \, ,
\end{split}
\end{equation}
%
where $\displaystyle{\left| \frac{d k}{d \omega} \right|}$ is the Jacobian 
for the transformation $k \to \omega$
and the integration is over all positive frequencies $\omega > 0$ excluding the frequency gaps 
of the band structure (see Figure \ref{fig:komega}).
We may rewrite Eq.\,(\ref{eq:flux_field_orig_omega1}) as
%
\begin{equation} \label{eq:flux_field_orig_omega2}
\begin{split}
    \hat{\Phi}_n(x,t) \, = \, \sqrt{\frac{\hbar}{2 C_0}} \, &
    \int_{0, \, \text{w/o gaps}}^{\infty} \, \frac{d\omega}{2 \pi} \, \left| \frac{d k}{d \omega} \right|^{1/2}
        \frac{1}{\sqrt{\omega}} \\[2mm]
    & \times \left[\,\hat{a}_n^{\,R}(\omega) \psi_{\omega}^{\,R}(x) \, e^{-i \omega \, t} \, + \,
    \hat{a}_n^{\,R \, \dagger}(\omega) \psi_{\omega}^{\,R \, *}(x) \, e^{i \omega \, t} \right. \\[2mm]
    & \quad \left. + \, \hat{a}_n^{\,L}(\omega) \psi_{\omega}^{\,L}(x) \, e^{-i \omega \, t} \, + \,
    \hat{a}_n^{\,L \, \dagger}(\omega) \psi_{\omega}^{\,L \, *}(x) \, e^{i \omega \, t} \, \right] \, \, ,
\end{split}
\end{equation}
%
where we defined (note that $k(\omega)>0$ by convention, see footnote \ref{foot:k} on page \pageref{foot:k})
%
\begin{subequations} \label{eq:defrl}
\begin{eqnarray}
& \hat{a}_n^{\,R}(\omega) := \displaystyle{\left| \frac{d k}{d \omega} \right|^{1/2}} \hat{a}_n\left[ \nu, k(\omega) \right] \, \, , 
\, \, \, \psi_{\omega}^{\,R}(x) := \psi_{\nu,\,k(\omega)}(x) \, \, , \, \, \,  \text{moving to the right (R)} \, \, , \\[2mm]
& \hat{a}_n^{\,L}(\omega) := \displaystyle{\left| \frac{d k}{d \omega} \right|^{1/2}} \hat{a}_n\left[ \nu, - k(\omega) \right] \, \, , 
\, \, \, \psi_{\omega}^{\,L}(x) := \psi_{\nu,\,-k(\omega)}(x) \, \, , \, \, \,  \text{moving to the left (L)} \, \, .
\end{eqnarray}
\end{subequations}
%
Note that by using $\omega$ as the independent variable, the band index $\nu$ is specified by the value of 
$\omega = \omega_{\,\nu}$ (see Figure \ref{fig:omegak}).
The prefactors of $\displaystyle{\left| \frac{d k}{d \omega} \right|^{1/2}}$ 
in Eq.\,(\ref{eq:defrl}) are introduced so that the commutation relations (\ref{eq:cra_orig}) 
for $\hat{a}_n(k)$ result in the corresponding commutation relations in $\omega$-space,
%
\begin{subequations} \label{eq:cra_omega}
\begin{eqnarray}
    & \left[ \hat{a}_n^{\,R}(\omega),{\hat a}_n^{\,R \, \dagger}(\omega') \right] = 2 \pi \delta(\omega - \omega') \, \, , \\[1mm]
    & \left[ \hat{a}_n^{\,L}(\omega),{\hat a}_n^{\,L \, \dagger}(\omega') \right] = 2 \pi \delta(\omega - \omega') \, \, , \\[1mm]
    & \left[ \hat{a}_n^{\,L}(\omega),{\hat a}_n^{\,R \, \dagger}(\omega') \right] = 0 \, \, , \, \, \text{etc.} \, \, ,
\end{eqnarray}
\end{subequations}
%
and $\hat{a}_n(\omega)$ has units of $\text{time}^{1/2}$.
\newpage
We now rewrite Eq.\,(\ref{eq:flux_field_orig_omega2}) in terms of unitless variables 
by expressing lengths in units of $\ell$ and times in units of $\ell/v$
as discussed in Section \ref{sec:BC_in_frequency_domain}:
%
\begin{equation} \label{eq:flux_field}
\begin{split}
    \hat{\phi}_n(x,t) \, = \, &
    \int_{0, \, \text{w/o gaps}}^{\infty} \, \frac{d\omega}{2 \pi} \, \left| \frac{d k}{d \omega} \right|^{1/2}
        \frac{1}{\sqrt{\omega}} \\[2mm]
    & \times \left[\, \hat{a}_n^{\,R}(\omega) \psi_{\omega}^{\,R}(x) \, e^{-i \omega \, t} \, + \,
    \hat{a}_n^{\,R \, \dagger}(\omega) \psi_{\omega}^{\,R \, *}(x) \, e^{i \omega \, t} \right. \\[2mm]
    & \quad \left. + \, \hat{a}_n^{\,L}(\omega) \psi_{\omega}^{\,L}(x) \, e^{-i \omega \, t} \, + \,
    \hat{a}_n^{\,L \, \dagger}(\omega) \psi_{\omega}^{\,L \, *}(x) \, e^{i \omega \, t} \, \right]
    \, \, , \quad n-1 < x < n \, \, ,
\end{split}
\end{equation}
%
with the unitless field operator in section $n$ of the periodic array 
(see Figure \ref{fig:system} on page \pageref{fig:system})
%
\begin{equation} \label{eq:ufo}
\hat{\phi}_n(x,t) := \sqrt{\frac{2 C_0 v}{\hbar}} \, \hat{\Phi}_n(x,t) \, \, , \quad n-1 < x < n \, \, .
\end{equation}
%
Unitless annihilation and creation operators in $\omega$-space are defined by 
$\displaystyle{\sqrt{\frac{v}{\ell}} \, \, \hat{a}_n(\omega)}$ 
and 
$\displaystyle{\sqrt{\frac{v}{\ell}} \, \, \hat{a}_n^\dagger(\omega)}$,
respectively, and again denoted by
%
$\hat{a}_n(\omega)$ and $\hat{a}_n^\dagger(\omega)$ to keep the notation simple. 
They obey the same commutation relations as in Eq.\,(\ref{eq:cra_omega}) (where $\omega$ is now unitless).
%
Note that all variables in Eq.\,(\ref{eq:flux_field}) are made unitless by expressing lengths in units of $\ell$ and times in units of $\ell/v$. The following results are then valid for any general system exhibiting time-dependent harmonic modulations in a periodic lattice. The variables that are unique to our superconducting circuit setup are the speed of propagation in the CPW $v$, the length of the CPW sections (lattice constant) $\ell$, 
and the unitless variable $\epsilon$ in Eq.\,(\ref{eq:epsilon2}), 
which determines the strength of the potential encountered at our SQUID sites. It follows that the subsequent analysis applies to any other physical realization with corresponding physical values of $\ell$, $v$, and $\epsilon$.

Equation (\ref{eq:flux_field}) incorporates our strategy to find the radiation created by the 
dynamical Casimir effect (DCE) in the SQUID array with time-dependent drive, Eq.\,(\ref{eq:energyexp1}). 
In each section $n$ of the array, the flux field $\hat{\phi}_n(x,t)$ for $n-1 < x < n$
is expanded in 
right-moving (R) and left-moving (L) Bloch waves $\psi_{\omega}^{\,R}(x)$ and $\psi_{\omega}^{\,L}(x)$ 
obtained for the static case (see Section \ref{subsec:kpsol}).
The expansion 
coefficients $\hat{a}_n^{\,R}(\omega)$ and $\hat{a}_n^{\,L}(\omega)$ 
in subsequent sections $n$, $n+1$ are coupled in a nontrivial way by the time-dependent 
part $\delta E_{J,\,n} \cos(\Omega \, t + \varphi_n)$ of the boundary condition at site $n$ in 
Eq.\,(\ref{eq:energyexp1}). 
The expansion coefficients in section $n+1$ are found from the coefficients in section $n$
from the boundary condition at site $n$ by using a transfer matrix approach, which allows 
us to find the coefficients and therefore the flux field for the whole array by iteration
(see Figure \ref{fig:system} on page \pageref{fig:system}). 


%%%%%%%%%%%%%%%%%%%%%%%%%%%%%%%%%%%%%%%%%%%%%%%%%%%%%%%%%%%%%%%%%%%%%%%%%%%%%%%%%%%%%%%%%%%%%%%%%%

\subsection{Frequency modes}
\label{subsec:frequency_modes}

\noindent
The next step will be to relate the operators $\hat{a}_n^{\,R}(\omega)$, $\hat{a}_n^{\,R \, \dagger}(\omega)$,
$\hat{a}_n^{\,L}(\omega)$, $\hat{a}_n^{\,L \, \dagger}(\omega)$ in Eq.\,(\ref{eq:flux_field}) 
in subsequent sections $n$, $n+1$ by using 
the time-dependent boundary condition (\ref{eq:BC_field_orig}) at SQUID site $n$.  
For a general function $f(t)$ of time $t$ we define its Fourier transform as 
%
\begin{equation} \label{eq:ftdef1}
\widetilde{f}(\omega) := \int\limits_{-\infty}^{\infty} dt \, f(t) \, \exp(i \omega \, t) 
\end{equation}  
%
and the inverse transformation
%
\begin{equation} \label{eq:ftdef2}
f(t) := \int\limits_{-\infty}^{\infty} \frac{d\omega}{2 \pi} \, \widetilde{f}(\omega) \, \exp(- i \omega \, t) \, \, .
\end{equation}  
%
Note that the integral in Eq.\,(\ref{eq:ftdef2}) includes both positive and negative frequencies $\omega$
whereas the integral in Eq.\,(\ref{eq:flux_field}) is restricted to positive frequencies $\omega>0$. 
However, one may rewrite the integral in Eq.\,(\ref{eq:flux_field}) to include 
both positive and negative frequencies $\omega$ by associating the terms containing the 
adjoint operators $\hat{a}_n^{\,R \, \dagger}(\omega)$ and $\hat{a}_n^{\,L \, \dagger}(\omega)$ 
with negative frequencies \cite{Johansson2010}:
%
\begin{equation} \label{eq:nf}
\hat{a}_n^{\,R}(-\omega) \psi_{-\omega}^{\,R}(x) \, e^{- i (- \omega) \, t} \, := \,
\hat{a}_n^{\,R \, \dagger}(\omega) \psi_{\omega}^{\,R \, *}(x) \, e^{i \omega \, t} \quad
\text{for} \, \, \omega > 0 \, \, ,
\end{equation}
%
which implies for $\omega<0$
%
\begin{equation} \label{eq:nf2}
\hat{a}_n^{\,R}(\omega) = \hat{a}_n^{\,R\,\dagger}(|\omega|) \, \, , \quad 
\psi_{\omega}^{\,R}(x) = \psi_{|\omega|}^{\,R \, *}(x) \quad
\text{for} \, \, \omega < 0 \, \, .
\end{equation}
%
Thus, positive frequencies $\omega>0$ are associated with operators $\hat{a}(\omega)$
and Bloch functions $\psi_{\omega}$ whereas negative frequencies $\omega<0$ are associated with
operators $\hat{a}^{\dagger}(|\omega|)$ and Bloch functions $\psi_{|\omega|}^*$
(compare Figure \ref{fig:jump} on page \pageref{fig:jump}).
Similar definitions apply for modes traveling to the left (L).
Using Eq.\,(\ref{eq:nf}) we can write
%
\begin{equation} \label{eq:ccterm}
\begin{split}
   & \int_{0, \, \text{w/o gaps}}^{\infty} \, \frac{d\omega}{2 \pi} \, \left| \frac{d k}{d \omega} \right|^{1/2}
    \frac{1}{\sqrt{\omega}} \, \, 
    \hat{a}_n^{\,R \, \dagger}(\omega) \psi_{\omega}^{\,R \, *}(x) \, e^{i \omega \, t} \\[4mm]
   = \, & \int_{-\infty, \, \text{w/o gaps}}^{0} \, \frac{d\omega}{2 \pi} \, \left| \frac{d k}{d \omega} \right|^{1/2}
    \frac{1}{\sqrt{|\omega|}} \, \, 
    \hat{a}_n^{\,R}(\omega) \psi_{\omega}^{\,R}(x) \, e^{- i \omega \, t}
\end{split}
\end{equation}
%
where the integral on the r.h.s.~is over the domain of negative frequencies
$\left\{ - \omega: \omega > 0, \, \text{w/o gaps} \right\}$, 
corresponding to a point reflection of the domain of the integral on the l.h.s.~at $\omega = 0$ 
to negative values. Using a similar definition for modes moving to the left ($L$), and using 
definition (\ref{eq:ftdef2}) for the Fourier transform, we may 
rewrite Eq.\,(\ref{eq:flux_field}) in compact form as
%
\begin{equation} \label{eq:flux_field2}
\begin{split}
    \hat{\phi}_n(x,t) \, & = \, 
    \displaystyle{
    \int_{-\infty, \, \text{w/o gaps}}^{\infty} \, \frac{d\omega}{2 \pi} \, \left| \frac{d k}{d \omega} \right|^{1/2}
        \frac{1}{\sqrt{|\omega|}}} \\[1mm]
    & \hspace{30mm} \times \left[ \, \hat{a}_n^{\,R}(\omega) \psi_{\omega}^{\,R}(x) \, + \, 
      \hat{a}_n^{\,L}(\omega) \psi_{\omega}^{\,L}(x)  \, \right] e^{-i \omega \, t} \\[4mm]
 & = \,  \int_{-\infty, \, \text{w/o gaps}}^{\infty} \, \frac{d\omega}{2 \pi} \,
     \hat{\phi}_n(x,\omega) \, e^{-i \omega \, t} \, \, ,
\end{split}
\end{equation}
%
where
%
\footnote{We omit the tilde symbol for the Fourier transform to the keep the notation simple.}
%
\begin{equation} \label{eq:flux_field3}
\hat{\phi}_n(x,\omega) \, = \, \hat{\phi}_n^{\,R}(x,\omega) \, + \, \hat{\phi}_n^{\,L}(x,\omega)
\end{equation}
%
with the frequency modes moving to the right ($R$)
and left ($T$) in section $n-1 < x < n$:
%
\begin{subequations} \label{eq:flux_field_omega}
\begin{eqnarray}
& \hat{\phi}_n^{\,R}(x,\omega) \, = \, \displaystyle{\left| \frac{d k}{d \omega} \right|^{1/2}
        \frac{1}{\sqrt{|\omega|}}} \, \, \hat{a}_n^{\,R}(\omega) \psi_{\omega}^{\,R}(x) \\[2mm]
& \, \, \hat{\phi}_n^{\,L}(x,\omega) \, = \, \displaystyle{\left| \frac{d k}{d \omega} \right|^{1/2}
        \frac{1}{\sqrt{|\omega|}}} \, \, \hat{a}_n^{\,L}(\omega) \psi_{\omega}^{\,L}(x) \, \, . 
\end{eqnarray}
\end{subequations}
%
We recall that the modes in Eq.\,(\ref{eq:flux_field_omega}) are defined for both positive and 
negative frequencies $\omega$, where negative frequencies $\omega<0$ produce the adjoint operator $\hat{a}^{\dagger}$ 
and complex conjugate $\psi^*$ as shown in Eq.\,(\ref{eq:nf2}). 
The frequency gaps in Eq.\,(\ref{eq:flux_field2}) can be reconciled with the definition 
(\ref{eq:ftdef2}) of the Fourier transform by setting the integrand in 
Eq.\,(\ref{eq:flux_field2}) to zero at values of $\omega$ that fall in frequency gaps. 


%%%%%%%%%%%%%%%%%%%%%%%%%%%%%%%%%%%%%%%%%%%%%%%%%%%%%%%%%%%%%%%%%%%%%%%%%%%%

\subsection{Incorporation of the boundary condition at the SQUID sites: Static case} 
\label{subsec:bc_static}

\noindent
In the static case, where  $\delta E_{J,\,n} = 0$ in Eq.\,(\ref{eq:energyexp1})
and the SQUID energy $E_{J,\,n} = E_J^0$ is constant, 
there are two boundary conditions for the frequency modes $\hat{\phi}_n(x,\omega)$ 
in Eq.\,(\ref{eq:flux_field3}) at SQUID site $n$ (see Section \ref{sec:kp} and 
Figure \ref{fig:system} on page \pageref{fig:system}):
%
\begin{enumerate}
    \item The (unitless) field operator $\hat{\phi}(x,\omega)$ is continuous at the 
    SQUID sites $n$:
    %
    \begin{equation} \label{eq:cont}
    \left. \hat{\phi}_n(x,\omega) \right|_{x=n} = 
    \left. \hat{\phi}_{n+1}(x,\omega) \right|_{x=n} \quad \text{for all integer} \, \, n \, \, .
    \end{equation}
    %
    \item The first derivative $\hat{\phi}'(x,\omega) := \partial \hat{\phi}(x,\omega) / \partial x$
    jumps at the SQUID sites $n$ 
    according to Eq.\,(\ref{eq:BC_freq_static_unitless_2}) with $\epsilon$ from 
    Eq.\,(\ref{eq:epsilon2}):
    %
    \begin{equation}\label{eq:jump}
    \epsilon \hat{\phi}_n(n, \omega) \, = \, \hat{\phi}_{n+1}'(n,\omega) \, - \, \hat{\phi}_n'(n,\omega) 
    \quad \text{for all integer} \, \, n \, \, .
    \end{equation}
    %
    This boundary condition couples $\hat{\phi}_n$ and $\hat{\phi}_{n+1}$
    because the derivative in the first term on the r.h.s.\,of Eq.\,(\ref{eq:jump}) 
    is taken in section $n+1$ (where $n < x < n+1$)
    whereas the derivative in the second term is taken in section $n$ (where $n-1 < x < n$)
    (see Figure \ref{fig:system} on page \pageref{fig:system}).
\end{enumerate}
%
Since by construction the Bloch functions $\psi_{\omega}^{\,R}(x)$, $\psi_{\omega}^{\,L}(x)$ 
solve the wave equation (\ref{eq:eom3}) for the flux field with the boundary conditions 
(\ref{eq:cont}), (\ref{eq:jump})
in the entire system (see Section \ref{subsec:kpsol}), in the static case 
the solution for the 
flux field $\hat{\phi}(x,\omega)$ in the entire system is simply a superposition of non-interacting Bloch modes:
%
\begin{equation} \label{eq:flux_field_static}
    \hat{\phi}(x,t) \, = \, 
    \displaystyle{
    \int_{-\infty, \, \text{w/o gaps}}^{\infty} \, \frac{d\omega}{2 \pi} \, \left| \frac{d k}{d \omega} \right|^{1/2}
        \frac{1}{\sqrt{|\omega|}}} \,
    \left[ \, \hat{a}^{\,R}(\omega) \psi_{\omega}^{\,R}(x) \, + \, 
    \hat{a}^{\,L}(\omega) \psi_{\omega}^{\,L}(x)  \, \right] e^{-i \omega \, t} \, \, ,
\end{equation}
%

\newpage
\noindent
where $\hat{a}^{\,R}(\omega)$, $\hat{a}^{\,L}(\omega)$ are now {\em global} coefficients 
(equal for all sections $n$ of the CPW) and
given by external conditions (e.g., radiation entering the system from outside, thermal radiation, etc.). 
For classical light, systems of this type are realized in photonic crystals \cite{Joannopoulos2008}, 
which consist of periodic dielectric nanostructures that affect the motion of photons in a similar way 
that periodic crystal lattices affect the motion of electrons in solids (see Section \ref{sec:kp})

Similar descriptions of quantum fields as an expansion of Bloch modes has been employed in the study of quantum electrodynamics in photonic crystals. Some recent examples are \cite{Gainutdinov2018}, where a QED treatment is used to study the Lamb shift of atoms placed in a photonic crystal medium, and \cite{Gainutdinov2021} where ionization energy of atoms in a photonic crystal is explored.

%%%%%%%%%%%%%%%%%%%%%%%%%%%%%%%%%%%%%%%%%%%%%%%%%%%%%%%%%%%%%%%%%%%%%%%%%%%%

\subsection{Incorporation of the boundary condition at the SQUID sites: Harmonic drive} 
\label{subsec:bc_drive}

\noindent
We now consider a time-dependent applied magnetic flux through the SQUIDs
in the form of a harmonic drive, which results in the SQUID energy 
$E_{J,\,n}(t) = E_J^0 + \delta E_{J,\,n} \cos(\Omega \, t + \varphi_n)$
in Eq.\,(\ref{eq:energyexp1}). 
From Eq.\,(\ref{eq:BC_field_orig}) and using unitless variables, we obtain the boundary 
condition for the (unitless) field operator $\hat{\phi}(x,t)$ introduced in Eq.\,(\ref{eq:ufo}): 
%
\begin{equation}\label{eq:BC_field_2}
\epsilon \hat{\phi}(n,t) \, + \, \delta \epsilon_n(t) \, \hat{\phi}(n,t) \, + \,
\hat{\phi}'(n^-,t) \, - \, \hat{\phi}'(n^+,t) \, = \, 0 
\quad \text{for all integer} \, \, n
\end{equation}
%
with 
$\epsilon = \ell \, \displaystyle{\left(\frac{2 \pi}{\Phi_{0}}\right)^{2}} E_J^0 \, L_0\,$
from Eq.\,(\ref{eq:epsilon2}), 
%
\begin{equation} \label{eq:deltaepsilon}
\delta \epsilon_n(t) \, = \, \delta \epsilon_n^0 \, \cos(\Omega \, t + \varphi_n)
\end{equation} 
%
with the unitless amplitude from Eq.\,(\ref{eq:deltaepsilon_alt})
%
\begin{equation} \label{eq:deltaepsilon2}
    \delta \epsilon_n^0 \, = \, \ell \left(\frac{2 \pi}{\Phi_{0}}\right)^{2} \,
\delta E_{J,\,n} \, L_0 \, \, ,
\end{equation}
%
and we omitted a term proportional to $C_J$ as discussed below
Eq.\,(\ref{eq:epsilon}).
\newpage
We now show how the time-dependent boundary condition (\ref{eq:BC_field_2}) results in a 
coupling between $\hat{a}$ operators in subsequent sections $n$, $n+1$
of the array. 
Fourier transforming the first, third and fourth terms in Eq.\,(\ref{eq:BC_field_2}) to 
frequency space according to Eq.\,(\ref{eq:ftdef1}) (for fixed positive $\omega>0$) 
results in frequency modes $\hat{\phi}_n(n,\omega)$,
$\hat{\phi}'_n(n,\omega)$, $\hat{\phi}'_{n+1}(n,\omega)$ as in the 
static case, with $\hat{\phi}_n(x,\omega)$ from Eq.\,(\ref{eq:flux_field3}). 
%
However, Fourier transforming the second term in Eq.\,(\ref{eq:BC_field_2}) 
(for fixed positive $\omega>0$) results in a convolution integral in frequency space:
%
\begin{equation} \label{eq:conv}
\int\limits_{-\infty}^{\infty} dt \, \delta \epsilon_n(t) \, \hat{\phi}_n(n,t) \, \exp(i \omega \, t) \, = \,
\int_{-\infty, \, \text{w/o gaps}}^{\infty} d\mu \, \sqrt{\frac{|\mu|}{|\omega|}} \, 
\delta g_n(\omega, \mu) \, \hat{\phi}_n(n,\mu)
\end{equation}
%
where
%
\begin{equation} \label{eq:deltag}
\delta g_n(\omega, \mu) \, := \, \frac{1}{2 \pi}
\sqrt{\frac{|\omega|}{|\mu|}} \int\limits_{-\infty}^{\infty} dt \, \delta \epsilon_n(t) \, 
\exp\left[i (\omega-\mu) t \right] \, \, .
\end{equation}
%
The factor of $\displaystyle{\frac{1}{2 \pi} \sqrt{\frac{|\omega|}{|\mu|}}}$ 
in Eq.\,(\ref{eq:deltag}) has been introduced for convenience.
Using Eq.\,(\ref{eq:conv}) the time-dependent boundary condition in 
Eq.\,(\ref{eq:BC_field_2}) reads in $\omega$-space
%
\begin{equation} \label{eq:bc4b}
\epsilon \hat{\phi}_n(n,\omega) \, + \,
\hat{\phi}'_n(n,\omega) \, - \, \hat{\phi}'_{n+1}(n,\omega) \, + \,
\int_{-\infty, \, \text{w/o gaps}}^{\infty} d\mu \, \sqrt{\frac{|\mu|}{|\omega|}} \, 
\delta g_n(\omega, \mu) \, \hat{\phi}_n(n,\mu) 
 \, = \, 0
\end{equation}
%
with $\hat{\phi}_n(x,\omega)$ from Eq.\,(\ref{eq:flux_field3}) and
the prime denotes a derivative $\displaystyle{\frac{\partial}{\partial x}}$ evaluated at $n$.
We now use the fact that the Bloch functions $\psi_{\omega}^{\,R}(x)$ and $\psi_{\omega}^{\,L}(x)$
in the definition of $\hat{\phi}_n(x,\omega)$ in 
Eqs.\,(\ref{eq:flux_field3}) and (\ref{eq:flux_field_omega}) 
fulfill the boundary condition in Eq.\,(\ref{eq:BC_freq_static_unitless_2}) for the static case, 
i.e., $\epsilon \psi_{\omega}(n) + \psi_{\omega}'(n^-) = \psi_{\omega}'(n^+)$. 
Using this in Eq.\,(\ref{eq:bc4b}) we find (for fixed positive $\omega>0$)
%
\begin{equation} \label{eq:bc4}
\begin{split}
& \left| \frac{dk}{d\omega} \right|^{1/2} \left[ \hat{a}_n^{\,R}(\omega) - \hat{a}_{n+1}^{\,R}(\omega) \right] 
\psi_{\omega}^{\,R\,'}(n^+) \, + \, 
\left| \frac{dk}{d\omega} \right|^{1/2} \left[ \hat{a}_n^{\,L}(\omega) - \hat{a}_{n+1}^{\,L}(\omega) \right] 
\psi_{\omega}^{\,L\,'}(n^+) \\[3mm]
& + \, \int_{-\infty, \, \text{w/o gaps}}^{\infty} d\mu \, \left| \frac{dk}{d\mu} \right|^{1/2} \,
\delta g_n(\omega, \mu) \, 
\left[ \hat{a}_n^{\,R}(\mu) \psi_{\mu}^{\,R}(n) \, + \, 
    \hat{a}_n^{\,L}(\mu) \psi_{\mu}^{\,L}(n) \right] \, = \, 0 \, \, .
\end{split}
\end{equation}
%
\newpage
Note that Eq.\,(\ref{eq:bc4}) couples operators $\hat{a}_n(\omega)$, $\hat{a}_{n+1}(\omega)$ in 
subsequent sections $n$, $n+1$ of the array in a nontrivial way due to the integral term. 
The derivative $\psi_{\omega}^{\,'}(n^+)$ of the Bloch functions is calculated at site $n$ from the right. 
In the following sections we devise a scheme to solve Eq.\,(\ref{eq:bc4}) numerically as 
a matrix equation for an harmonic drive 
$\delta \epsilon_n(t) \, = \, \delta \epsilon_n^0 \, \cos(\Omega \, t + \varphi_n)$
in Eq.\,(\ref{eq:deltag}).


%%%%%%%%%%%%%%%%%%%%%%%%%%%%%%%%%%%%%%%%%%%%

\subsubsection{Bloch functions at the SQUID sites} \label{subsubsec:bloch_functions}
%
We recall that the Bloch functions in Eq.\,(\ref{eq:bc4}) are given in terms of the wave vector $k(\omega)$
by Eq.\,(\ref{eq:defrl}), where we defined $k(\omega)>0$ by convention 
(see Eq.\,(\ref{eq:disp_rel_k}) and footnote \ref{foot:k} on page \pageref{foot:k}). 
To avoid confusion regarding the sign of $k$ in future calculations, let us define
explicitly 
%
\begin{equation} \label{eq:kdef}
K := |k| > 0 \, \, \, , \quad K(\omega) := |k(\omega)| > 0 \, \, ,
\end{equation}
%
so that $k = K$ if $k>0$ and $k = - K$ if $k<0$. We thus obtain for $\omega>0$
%
\footnote{To keep the notation simple, in this section we omit the band index $\nu$. It is 
understood that a frequency band is uniquely specified by $k$ and $\nu$, or by 
$\omega = \omega_{\,\nu}$. See Eq.\,(\ref{eq:defrl}) in Section \ref{subsec:sq}.}
%
\begin{subequations} \label{eq:bloch3}
\begin{eqnarray}
& \psi_{\omega}^{\,R}(x) = \psi_{K(\omega)}(x) = \exp\left[i K(\omega) x\right] \, u_{K(\omega)}(x)
\, \, , \quad \text{moving to the right (R)} \, \, , \\[2mm]
& \psi_{\omega}^{\,L}(x) = \psi_{-K(\omega)}(x) = \exp\left[- i K(\omega) x\right] \, u_{K(\omega)}^*(x)
\, \, , \quad \text{moving to the left (L)} \, \, ,
\end{eqnarray}
\end{subequations}
%
where in the second line we used $u_{-K}(x) = u_K^*(x)$. For $\omega<0$ we obtain, 
using Eq.\,(\ref{eq:nf2}),
%
\begin{subequations} \label{eq:bloch3_neg}
\begin{eqnarray}
& \psi_{\omega}^{\,R}(x) = \psi_{|\omega|}^{\,R \, *}(x)
= \exp\left[- i K(\omega) x \right] \, u_{K(\omega)}^*(x) 
\, \, , \quad \text{moving to the right (R)} \, \, , \\[2mm]
& \psi_{\omega}^{\,L}(x) = \psi_{|\omega|}^{\,L \, *}(x)
= \exp\left[i K(\omega) x \right] \, u_{K(\omega)}(x) 
\, \, , \quad \text{moving to the left (L)} \, \, .
\end{eqnarray}
\end{subequations}
%
In Eq.\,(\ref{eq:bc4}) we need the values of the Bloch functions $\psi(x)$ and their derivatives at the SQUID sites $n$.  
Since $\psi(x) = \exp(i K x) u_K(x)$ and 
$\psi'(x) = \exp(i K x) \left[i K u_K(x) + u_K'(x)\right]$ we define
\newpage
%
\begin{subequations} \label{eq:cal}
\begin{eqnarray}
& {\cal C}_K := u_K(0) = u_K(1) =  u_{-K}(0) = u_{-K}(1) \quad \text{real-valued} \\[2mm]
& {\cal D}_K := \displaystyle{\frac{d}{dx}} \left. u_K(x) \right|_{x=0^+} \, \, \, , \label{eq:cald}
\quad {\cal D}_{-K} = {\cal D}_K^* \, \, , \\[2mm]
& {\cal A}_K := i K {\cal C}_K + {\cal D}_K \, \, \, , \quad 
{\cal A}_{-K} = {\cal A}_K^* \, \, .
\end{eqnarray}
\end{subequations}
%
We define for $\omega>0$
%
\begin{equation} \label{eq:calomega}
{\cal C}_{\omega} := {\cal C}_{K(\omega)} \, \, , \quad 
{\cal D}_{\omega} := {\cal D}_{K(\omega)} \, \, , \quad 
{\cal A}_{\omega} := {\cal A}_{K(\omega)} \, \, ,
\end{equation}
%
whereas for $\omega<0$
%
\begin{equation} \label{eq:calomega2}
{\cal C}_{\omega} = {\cal C}_{|\omega|} \, \, , \quad 
{\cal D}_{\omega} = {\cal D}_{|\omega|}^* \, \, , \quad 
{\cal A}_{\omega} = {\cal A}_{|\omega|}^* \, \, .
\end{equation}
%
Note that $u_K(x)$ is defined in the domain $x \in [0,1]$ so that the derivative in Eq.\,(\ref{eq:cald})
is taken for $x>0$ and evaluated at $x=0$. The algebraic properties shown in Eq.\,(\ref{eq:cal})
are verified in the Mathematica notebook in the Appendix. %\ref{ch:Mathematica_KP}.
With these definitions, the Bloch functions and their derivatives at site $n$ in Eq.\,(\ref{eq:bc4}) are given by, for $\omega>0$,
%
\begin{equation} \label{eq:bloch_n}
\begin{split}
& \psi_{\omega}^{\,R}(n) = \exp\left[i K(\omega) n\right] \, {\cal C}_{\omega} \, \, \, , \quad
\psi_{\omega}^{\,L}(n) = \exp\left[- i K(\omega) n\right] \, {\cal C}_{\omega} \, \, , \\[2mm]
& \psi_{\omega}^{\,R\,'}(n^+) = \exp\left[i K(\omega) n\right] \, {\cal A}_{\omega} \, \, \, , \quad
\psi_{\omega}^{\,L\,'}(n^+) = \exp\left[- i K(\omega) n\right] \, {\cal A}_{\omega}^* \, \, , 
\end{split}
\end{equation}
%
whereas for $\omega<0$ 
%
\begin{equation} \label{eq:bloch_n_neg}
\begin{split}
& \psi_{\omega}^{\,R}(n) = \exp\left[- i K(\omega) n\right] \, {\cal C}_{|\omega|} \, \, \, , \quad
\psi_{\omega}^{\,L}(n) = \exp\left[i K(\omega) n\right] \, {\cal C}_{|\omega|} \, \, , \\[2mm]
& \psi_{\omega}^{\,R\,'}(n^+) = \exp\left[-i K(\omega) n\right] \, {\cal A}_{|\omega|}^* \, \, \, , \quad
\psi_{\omega}^{\,L\,'}(n^+) = \exp\left[i K(\omega) n\right] \, {\cal A}_{|\omega|} \, \, . 
\end{split}
\end{equation}
%



%%%%%%%%%%%%%%%%%%%%%%%%%%%%%%%%%%%%%%

\subsubsection{Position-dependent annihilation and creation operators} \label{subsubsec:annihilation_creation}
%
From Eqs.\,(\ref{eq:bc4}) and (\ref{eq:bloch_n}) one can see that the 
operators $\hat{a}_n^{\,R}$ are associated with a phase factor $\exp(i K n)$ and 
operators $\hat{a}_n^{\,L}$ are associated with a phase factor $\exp(- i K n)$,
where the phase factors come from the Bloch functions.
In future calculations it will be convenient to include this phase factor in the 
operators by defining position-dependent operators (for positive $\omega>0$)
%
\begin{subequations} \label{eq:defrl_2}
\begin{eqnarray}
& \hat{a}_n^{\,R}(\omega,x) := \hat{a}_n^{\,R}(\omega) \exp\left[i K(\omega) x\right]  
\, \, , \quad n-1 < x < n \, \, , \\[2mm]
& \hat{a}_n^{\,L}(\omega,x) := \hat{a}_n^{\,L}(\omega) \exp\left[-i K(\omega) x\right] 
\, \, , \quad n-1 < x < n \, \, ,
\end{eqnarray}
\end{subequations}
%
with $\hat{a}_n^{\,R}(\omega)$, $\hat{a}_n^{\,L}(\omega)$ as defined in Eq.\,(\ref{eq:defrl})
%
\footnote{The phase factor in Eq.\,(\ref{eq:defrl_2}) does not change the 
commutation relations (\ref{eq:cra_omega}).}. 
%
Using the above definitions, Eq.\,(\ref{eq:bc4}) takes the form
%
\begin{equation} \label{eq:bc5}
\begin{split}
& \left| \frac{dk}{d\omega} \right|^{1/2} \left[ \hat{a}_n^{\,R}(\omega,n) - \hat{a}_{n+1}^{\,R}(\omega,n) \right] 
{\cal A}_{\omega} \, + \, 
\left| \frac{dk}{d\omega} \right|^{1/2} \left[ \hat{a}_n^{\,L}(\omega,n) - \hat{a}_{n+1}^{\,L}(\omega,n) \right] 
{\cal A}_{\omega}^* \\[3mm]
& + \, \int_{-\infty, \, \text{w/o gaps}}^{\infty} d\mu \, \left| \frac{dk}{d\mu} \right|^{1/2} \,
\delta g_n(\omega, \mu) \, {\cal C}_{\mu} 
\left[ \hat{a}_n^{\,R}(\mu,n) \, + \, \hat{a}_n^{\,L}(\mu,n) \right] \, = \, 0 \, \, .
\end{split}
\end{equation}
%




%%%%%%%%%%%%%%%%%%%%%%%%%%%%%%%%%%%%%%

\subsubsection{Implementation of the harmonic drive} \label{subsubsec:harmonic_drive}

We now specify the kernel $\delta g_n(\omega, \mu)$ in Eq.\,(\ref{eq:bc5}) for the harmonic drive $\delta \epsilon_n(t) = \delta \epsilon_n^0 \, \cos(\Omega \, t + \varphi_n)$
in Eq.\,(\ref{eq:deltaepsilon}). Using this form in Eq.\,(\ref{eq:deltag}) we find
%
\begin{equation} \label{eq:deltag_harm}
\delta g_n(\omega, \mu) \, = \, \sqrt{\frac{|\omega|}{|\mu|}} \, \frac{\delta \epsilon_n^0}{2} 
\left\{
e^{i \varphi_n} \, \delta\left[\mu - (\omega + \Omega)\right] \, + \,  
e^{- i \varphi_n} \, \delta\left[\mu - (\omega - \Omega)\right]
\right\} \, \, ,
\end{equation}
%
which implies that the frequency $\omega$ is scattered to $\omega \pm \Omega$
provided that the target frequencies $\omega \pm \Omega$ are allowed, i.e., 
do not fall in a frequency gap of the band structure
(see Figure \ref{fig:jump} on page \pageref{fig:jump}). 
Thus, for given $\omega$, the range of frequencies resulting from multiple scattering 
processes is confined to the discrete set 
$\left\{\omega + \alpha \, \Omega \, \, \, \text{for integer} \, \, \alpha \right\}$. 
For our numerical implementation we need to limit this set to a finite number 
of elements. Taking into account that the range of frequencies can only include 
allowed frequencies, for given (allowed) frequency $\omega$ we define the range of 
frequencies due to multiple scattering processes as
%
\begin{equation} \label{eq:omega_range}
\omega_{\,\alpha} = \omega + \left( N_u + 1 - \alpha \right) \Omega \, \, , \quad \alpha = 1, \ldots, M(\omega)
\end{equation}
%
where 
%
\begin{equation} \label{eq:def_m}
M(\omega) = N_u(\omega) + N_l(\omega) + 1 \, \, .
\end{equation}
%
Explicitly, the range of frequencies in Eq.\,(\ref{eq:omega_range}) is given by 
%
\begin{equation} \label{eq:table}
\begin{array}{l}
\omega_{\,1} = \omega + N_u \, \Omega \\
\omega_{\,2} = \omega + (N_u - 1) \, \Omega \\
\vdots \\
\omega_{\,\alpha} = \omega \, \, \, \, \text{with} \, \, \, \alpha = N_u + 1 \\
\vdots \\
\omega_{\,M-1} = \omega - (N_l - 1) \, \Omega \\
\omega_{\,M} = \omega - N_l \, \Omega \, \, .
\end{array}
\end{equation}
%
The upper and lower boundaries $N_u$, $N_l$ are  defined as follows. 
For given frequency $\omega$, $N_u(\omega)$ and $N_l(\omega)$
are the {\em maximum} integers $\in \left\{0, \ldots, N_{cut} \right\}$ so that {\em all} frequencies
$\omega_{\,1}, \ldots, \omega_{\,M}$ in Eq.\,(\ref{eq:table}) are allowed. 
Thus, $\omega_{\,1}$ is allowed but $\omega_{\,1} + \Omega$ 
is not allowed, i.e., falls in a frequency gap. Similarly,
$\omega_{\,M}$ is allowed but $\omega_{\,M} - \Omega$ is not allowed
%
\footnote{We may call the set of frequencies in Eq.\,(\ref{eq:table}) the {\em largest 
simply connected frequency domain} that includes $\omega$.}.
%
$N_{cut}$ is an overall cutoff that limits $N_u$, $N_l$ if the scattered frequencies 
do not fall in gaps within the range specified by $N_{cut}$. 
Note that $N_u(\omega)$, $N_l(\omega)$, $M(\omega)$ in Eq.\,(\ref{eq:def_m}) 
depend on the given frequency $\omega$, whereas $N_{cut}$ is a global cutoff 
and independent of $\omega$. 

Multiple scattering processes may lead from a positive initial frequency 
$\omega = \omega_{\,\alpha}$ (with $\alpha = N_u + 1$, see Eq.\,(\ref{eq:table}))
to negative frequencies $\omega_{\,\beta} < 0$. 
Negative frequencies turn creation and annihilation operators into their conjugate 
versions as shown in Eq.\,(\ref{eq:nf}). It is this crossover from positive to 
negative frequencies and the associated conjugation of 
creation and annihilation operators that causes the DCE radiation, 
as will be shown in Section \ref{sec:dcr} below. For a graphical representation 
of the frequency jumps generated by these scattering processes, see Figure \ref{fig:jump}.
%

\begin{figure}
    \includegraphics[width=1.0\textwidth, keepaspectratio]{figures/system/jump.png}
    \caption{Band structure with frequency gaps (shown shaded) and possible transitions induced by an harmonic drive with 
    frequency $\Omega$ (see Eq.\,(\ref{eq:omega_range})). 
    The harmonic drive induces jumps from an allowed frequency $\omega$ in units of the drive frequency $\Omega$ (blue, 
    here $\Omega=8$)
    if the target frequencies $\omega \pm \Omega$ are allowed, i.e., do not fall in a band gap (green). 
    In this figure the domain of positive frequencies $\omega>0$
    is extended to negative frequencies $\omega<0$ by a point reflection at $\omega=0$
    (see text below Eq.\,(\ref{eq:ccterm}) and compare Figure\,\ref{fig:omegak} on page \pageref{fig:omegak}). 
    Physical radiation corresponds to positive frequencies $\omega > 0$. An operator $\hat{a}$ for a 
    negative frequency $\omega - \Omega < 0$ thus transforms to the adjoint operator $\hat{a}^{\dagger}$ 
    for positive frequency $|\omega - \Omega| > 0$ (magenta) (see Eqs.\,(\ref{eq:nf}) and \ref{eq:nf2})). 
    This process generates DCE radiation by virtue of the commutation relation 
    $\left[ \hat{a}, \hat{a}^{\dagger} \right] = 1$ (see Eq.\,(\ref{eq:c5})).  
    In general, this type of process is described by a Bogoliubov transformation 
    (see Section \ref{sec:ZPE_and_DCE}).} 
    \label{fig:jump}
\end{figure}

If for given $\omega$ a transition from positive to negative frequencies occurs within the range specified by Eq.\,(\ref{eq:table}), we use the special index $\kappa$ for the {\em first} negative frequency $\omega_{\kappa}<0$ found by reading the table (\ref{eq:table}) from top to bottom. 
Thus
%
\begin{equation} \label{eq:table_kappa}
\begin{array}{l}
\omega_{\,1} > 0 \\
\vdots \\
\omega_{\,\kappa-1} > 0 \\
\omega_{\,\kappa} < 0 \\
\vdots \\
\omega_{\,M} < 0 \, \, \, .
\end{array}
\end{equation}

\noindent
Using the definitions above, denoting $\hat{a}_n^{\,R}(\alpha,n) \equiv \hat{a}_n^{\,R}(\omega_{\alpha},n)$, etc., 
and using the abbreviation
%
\begin{equation} \label{eq:dwn}
d\omega(\alpha) := \left|\frac{dk}{d\omega_{\,\alpha}} \right|^{1/2} \, \, ,
\end{equation}
%
the boundary condition (\ref{eq:bc5}) at SQUID site $n$ for given frequency $\omega = \omega_{\alpha}$ reads
%
\begin{equation} \label{eq:bc6}
\begin{split}
& d\omega(\alpha) \left[ \hat{a}_n^{\,R}(\alpha,n) - \hat{a}_{n+1}^{\,R}(\alpha,n) \right] 
{\cal A}_{\alpha} \, + \, 
d\omega(\alpha) \left[ \hat{a}_n^{\,L}(\alpha,n) - \hat{a}_{n+1}^{\,L}(\alpha,n) \right] 
{\cal A}_{\alpha}^* \\[3mm]
& + \, \frac{\delta \epsilon_n^0}{2}
\sum_{\beta, \, \text{w/o gaps}} d\omega(\beta) \, 
\sqrt{\frac{|\omega_{\alpha}|}{|\omega_{\beta}|}} \, {\cal C}_{\beta} \,
\left[e^{i \varphi_n} \, \delta_{\,\beta,\,\alpha+1} + e^{- i \varphi_n} \, \delta_{\,\beta,\,\alpha-1} \right]
\left[ \hat{a}_n^{\,R}(\beta,n) \, + \, \hat{a}_n^{\,L}(\beta,n) \right] \, = \, 0 \, \, .
\end{split}
\end{equation}

%%%%%%%%%%%%%%%%%%%%%%%%%%%%%%%%%%%%%%%%%%%%%%%%

\subsubsection{Vector notation and transfer matrix} \label{subsubsec:vector}

To write Eq.\,(\ref{eq:bc6}) in matrix form, for given $\omega$ with largest simply connected 
domain given by Eq.\,(\ref{eq:table})  
we divide Eq.\,(\ref{eq:bc6}) by $\epsilon$ 
from Eq.\,(\ref{eq:epsilon2}) and define matrices $\delta G^{\,(n)}$ and $A$ by
%
\footnote{Note that Greek indices $\alpha$, $\beta$, etc.~label the set of frequencies 
$\omega_{\,\alpha}$ in Eq.\,(\ref{eq:table}) whereas Latin indices $n$ label the SQUID sites.}
%
\begin{equation} \label{eq:matrix_gtilde}
\delta G_{\alpha \beta}^{\,(n)} \, := \, \frac{1}{2} \frac{\delta \epsilon_n^0}{\epsilon} \, 
d\omega(\beta) \,
\sqrt{\frac{|\omega_{\alpha}|}{|\omega_{\beta}|}} \, {\cal C}_{\beta} \,
\left[e^{i \varphi_n} \, \delta_{\,\beta,\,\alpha+1} + e^{- i \varphi_n} \, \delta_{\,\beta,\,\alpha-1} \right] \, \, ,
\end{equation}
%
\begin{equation} \label{eq:matrix_a}
A_{\alpha \beta} \, := \, \frac{1}{\epsilon} \, d\omega(\alpha) \, \delta_{\alpha \beta} \, \times \,
\left\{
\begin{array}{l}
{\cal{A}}_{\alpha} \, \, \, , \, \, \omega_{\,\alpha}>0  \\
{\cal{A}}_{\alpha}^* \, \, \, , \, \, \omega_{\,\alpha}<0 \, \, \, .
\end{array} \right.
\end{equation}
%
Note that the matrix $\delta G^{\,(n)}$ in general depends on the SQUID site $n$ in terms
of the amplitude $\epsilon_n^0$ and phase $\varphi_n$ in 
$\delta \epsilon_n(t) = \delta \epsilon_n^0 \, \cos(\Omega \, t + \varphi_n)$
in Eq.\,(\ref{eq:deltaepsilon}).
%
Furthermore, we combine annihilation operators for the frequencies in Eq.\,(\ref{eq:table}) to a vector
%
\begin{equation} \label{eq:vector_x}
\vec{X}_R \, = \, 
\begin{pmatrix}
\hat{a}_n^{\,R}(\omega_{\,1},n) \\
\vdots \\
\hat{a}_n^{\,R}(\omega_{\,\kappa-1},n) \\
\hat{a}_n^{\,R}(\omega_{\,\kappa},n) \\
\vdots \\
\hat{a}_n^{\,R}(\omega_{\,M},n) \\
\end{pmatrix}
\, = \,
\begin{pmatrix}
\hat{a}_n^{\,R}(\omega_{\,1},n) \\
\vdots \\
\hat{a}_n^{\,R}(\omega_{\,\kappa-1},n) \\
\hat{a}_n^{\,R\,\dagger}(|\omega_{\,\kappa}|,n) \\
\vdots \\
\hat{a}_n^{\,R\,\dagger}(|\omega_{\,M}|,n) \\
\end{pmatrix}
\end{equation}
%
where the index $\kappa$ is defined in Eq.\,(\ref{eq:table_kappa}) and for the second equation we used
Eq.\,(\ref{eq:nf2}). 
Note that the vector $\vec{X}_R$ is defined for section $n$ of the SQUID array at position $x=n$, i.e.,
right at and to the left of SQUID $n$ (see Figure \ref{fig:xy}).
%
To simplify notation in the following discussion, we omit the index $n$ for $\vec{X}_R$ and it is 
understood that $\vec{X}_R$ is an operator. 
In a completely analogous way as in Eq.\,(\ref{eq:vector_x}) we define a vector $\vec{X}_L$ for 
left-moving modes (L). 

In a similar way we define vectors $\vec{Y}_R$, $\vec{Y}_L$
for section $n+1$ of the SQUID array at position $x=n$, i.e., right at and to the right of SQUID $n$ 
(see Figure \ref{fig:xy}):
%
\begin{equation} \label{eq:vector_y}
\vec{Y}_R \, = \, 
\begin{pmatrix}
\hat{a}_{n+1}^{\,R}(\omega_{\,1},n) \\
\vdots \\
\hat{a}_{n+1}^{\,R}(\omega_{\,\kappa-1},n) \\
\hat{a}_{n+1}^{\,R}(\omega_{\,\kappa},n) \\
\vdots \\
\hat{a}_{n+1}^{\,R}(\omega_{\,M},n) \\
\end{pmatrix}
\, = \,
\begin{pmatrix}
\hat{a}_{n+1}^{\,R}(\omega_{\,1},n) \\
\vdots \\
\hat{a}_{n+1}^{\,R}(\omega_{\,\kappa-1},n) \\
\hat{a}_{n+1}^{\,R\,\dagger}(|\omega_{\,\kappa}|,n) \\
\vdots \\
\hat{a}_{n+1}^{\,R\,\dagger}(|\omega_{\,M}|,n) \\
\end{pmatrix}
\end{equation}
%
and similarly for $\vec{Y}_L$. With these definitions, Eq.\,(\ref{eq:bc6}) becomes a matrix equation
(omitting the index $n$ for the matrix $\delta G$)
%
\begin{equation} \label{eq:bc7}
A \left(\vec{X}_R - \vec{Y}_R \right) + A^* \left(\vec{X}_L - \vec{Y}_L \right) + 
\delta G \left(\vec{X}_R + \vec{X}_L \right) = 0 \, \, ,
\end{equation} 
%
which can be solved for the $\vec{Y}$ vectors as
%
\begin{equation} \label{eq:bc8}
A \vec{Y}_R + A^* \vec{Y}_L =
A \vec{X}_R + A^* \vec{X}_L + \delta G \left(\vec{X}_R + \vec{X}_L \right) \, \, .
\end{equation}
% 
The condition that the flux field $\hat{\phi}(x,\omega)$ in Eq.\,(\ref{eq:flux_field_static})
is continuous at the SQUID sites $n$ gives a second boundary condition in matrix form, 
using Eqs.\,(\ref{eq:bloch_n}) and (\ref{eq:defrl_2}) (where a factor ${\cal C}_{\omega}$ cancels)
%
\begin{equation} \label{eq:bc_cont_matrix}
\vec{Y}_R + \vec{Y}_L = \vec{X}_R + \vec{X}_L \, \, .
\end{equation}
%
The two equations (\ref{eq:bc8}), (\ref{eq:bc_cont_matrix}) can be solved $\vec{Y}_R$, $\vec{Y}_L$
in a straightforward way. The solution can be given in a compact form as follows. 
We combine the vectors for right-moving (R) and left-moving (L) modes to $2M$-vectors as
%
\begin{equation} \label{eq:x_comp}
\underline{X} = 
\begin{pmatrix}
\vec{X}_R \\
\vec{X}_L \\
\end{pmatrix} \, \, \, , \quad
\underline{Y} = 
\begin{pmatrix}
\vec{Y}_R \\
\vec{Y}_L \\
\end{pmatrix} \, \, \, ,
\end{equation}
%
and define a $2M \times 2M$ matrix $S$ as
%
%
\begin{equation} \label{eq:matrix_s}
S =
\begin{pmatrix}
I + {\cal G} & {\cal G} \\
-{\cal G} & I - {\cal G} \\
\end{pmatrix} \, \, \, ,
\end{equation}
%
where $I$ is the identity matrix and the $M \times M$ matrix ${\cal G}$ is defined as
%
\begin{equation} \label{eq:matrix_G}
{\cal G} = \left( A - A^* \right)^{-1} \delta G \, \, .
\end{equation}
%
The matrix $\left( A - A^* \right)^{-1}$ is diagonal with purely imaginary diagonal elements 
$\displaystyle{\frac{\epsilon \delta_{\alpha \beta}}{d \omega_{\,\alpha} \left({\cal{A}}_{\alpha}
- {\cal{A}}_{\alpha}^* \right)}}$
where \newline
${\cal A}_{\alpha} - {\cal A}_{\alpha}^* = \pm 2 i \left[ K(\omega_{\,\alpha}) {\cal C}_{\alpha} 
+ \text{Im}( {\cal D}_{\alpha}) \right]$ for $\omega_{\,\alpha} > 0$ ($+$) or $<0$ ($-$) 
(see Eq.\,(\ref{eq:matrix_a})). 
%
The solution for the $\vec{Y}$ vectors takes the compact form (see Figure \ref{fig:xy})
%
\begin{equation} \label{eq:matrix_compact}
    \underline{Y} = S \, \underline{X} \, \, ,
\end{equation}
%
where both vectors are calculated at $x=n$ (but on opposite sides).
Note that in the static case, where ${\cal G} = \delta G = \delta \epsilon_n^0 = 0$,
the matrix
$S$ in Eq.\,(\ref{eq:matrix_s}) reduces to the $2M \times 2M$ identity matrix,
so that $\underline{Y}(x=n) = \underline{X}(x=n)$ for all SQUID sites $n$. 
This implies that the operators $\vec{a}_n$, $\vec{a}_{n+1}$ 
in the $\vec{X}$, $\vec{Y}$ vectors in Eqs.\,(\ref{eq:vector_x}), (\ref{eq:vector_y})
are equal for all $n$, and we recover the result in Eq.\,(\ref{eq:flux_field_static})
found in the static case. 

\begin{figure}
    %
    \includegraphics[width=0.4\textwidth, keepaspectratio]{figures/system/xy.png}
    \caption{Matrix $S$ transforming vectors $\vec{X}_R$, $\vec{X}_L$ on the left of SQUID site $n$
    to vectors $\vec{Y}_R$, $\vec{Y}_L$ on the right of SQUID site $n$ according to 
    Eq.\,(\ref{eq:matrix_compact}). The directions of propagation of the associated operators  
    are indicated by the blue arrows.}
    \label{fig:xy}
\end{figure}

According to the definitions (\ref{eq:vector_x}), (\ref{eq:vector_y}), the matrix $S$ 
in Eq.\,(\ref{eq:matrix_compact}) maps operators $\hat{a}_n(\alpha,n)$ in  
section $n$ of the SQUID array at position $x=n$ to operators $\hat{a}_{n+1}(\alpha,n)$ in
section $n+1$ at position $x=n$, i.e., "transfers" the operators across the SQUID site $n$. 
In the next step we transfer the latter operators within section $n+1$ from position $x=n$
to $x=n+1$, i.e, from SQUIDs $n$ to $n+1$. According to Eq.\,(\ref{eq:defrl_2}) this
transition is accompanied by phase factors 
%
\begin{equation} \label{eq:vector_y2}
\vec{Y}_R(x=n+1) \, = \, 
\begin{pmatrix}
\hat{a}_{n+1}^{\,R}(\omega_{\,1},n+1) \\
\vdots \\
\hat{a}_{n+1}^{\,R}(\omega_{\,\kappa-1},n+1) \\
\hat{a}_{n+1}^{\,R\,\dagger}(|\omega_{\,\kappa}|,n+1) \\
\vdots \\
\hat{a}_{n+1}^{\,R\,\dagger}(|\omega_{\,M}|,n+1) \\
\end{pmatrix}
\, = \, 
\begin{pmatrix}
\hat{a}_{n+1}^{\,R}(\omega_{\,1},n) \, \exp\left[i K(\omega_{\,1}) \right] \\
\vdots \\
\hat{a}_{n+1}^{\,R}(\omega_{\,\kappa-1},n) \, \exp\left[i K(\omega_{\,\kappa-1}) \right] \\
\hat{a}_{n+1}^{\,R\,\dagger}(|\omega_{\,\kappa}|,n) \, \exp\left[- i K(\omega_{\,\kappa}) \right] \\
\vdots \\
\hat{a}_{n+1}^{\,R\,\dagger}(|\omega_{\,M}|,n) \, \exp\left[- i K(\omega_{\,M}) \right] \\
\end{pmatrix}
\, = \,
{\cal P}(1) \, \vec{Y}_R(x=n)
\end{equation}
%
with the $M \times M$ diagonal matrix
%
\begin{equation} \label{eq:matrix_p}
{\cal P}(x) \, = \, \text{diag}\left(\exp\left[i K(\omega_{\,1}) x \right], \ldots, 
\exp\left[i K(\omega_{\,\kappa-1}) x \right],
\exp\left[- i K(\omega_{\,\kappa}) x  \right], \ldots,
\exp\left[- i K(\omega_{\,M}) x \right] \right) \, \, .
\end{equation}
%
Similarly we obtain $\vec{Y}_L(x=n+1) = {\cal P}^*(1) \, \vec{Y}_L(x=n)$. 
Defining the $2M \times 2M$ matrix $P$ by
%
\begin{equation} \label{eq:matrix_p2}
P(x) \, = \, 
\begin{pmatrix}
{\cal P}(x) & 0 \\
0 & {\cal P}^*(x) \\
\end{pmatrix}
\end{equation}
%
we thus obtain
%
\begin{equation} \label{eq:matrix_ypy}
\underline{Y}(x=n+1) \, = \, P(1) \, \underline Y(x=n) \, \, .
\end{equation}
%
Thus, the matrix $P(1)$ transfers operators $\hat{a}_{n+1}(\alpha,n)$ in
section $n+1$ at position $x=n$ to position $n+1$ corresponding to a displacement of 
1 lattice constant $\ell$. 
Combining Eqs.\,(\ref{eq:matrix_compact}) and (\ref{eq:matrix_p2}) 
we finally obtain
%
\begin{equation} \label{eq:matrix_t}
\underline{Y}_{\, n+1}(x=n+1) \, = \, P(1) \, S_n \, \underline{X}_{\,n}(x=n) \, =: \, T_n \, \underline{X}_{\,n}(x=n) \, \, ,
\end{equation}
%
with the transfer matrix 
%
\begin{equation} \label{eq:matrix_t2}
T_n = P(1) \, S_n \, \, .
\end{equation}
%
In Eq.\,(\ref{eq:matrix_t}) we reintroduce indices to indicate that $\underline{Y}_{\, n+1}(x=n+1)$ contains operators in section $n+1$ evaluated at $x=n+1$
and $\underline{X}_{\,n}(x=n)$
contains operators in section $n$ evaluated at $x=n$. 
The transfer matrix $T_n$ in Eq.\,(\ref{eq:matrix_t2}) thus transfers the latter 
to the right by one repeated unit of the periodic SQUID array (see Figure \ref{fig:system}).
%
Furthermore, we use the index $n$ for $T_n$ to indicate that the transfer matrix
depends on the SQUID site $n$ by the contribution $S = S_n$ in Eq.\,(\ref{eq:matrix_s}) 
(see text below Eq.\,(\ref{eq:matrix_a})).  
Finally, for given frequency $\omega$, both $S_n$ and $P(1)$ in Eq.\,(\ref{eq:matrix_t2}) 
depend on the set of frequencies $\omega_{\,\alpha}$ in Eq.\,(\ref{eq:table}) by the inverse 
dispersion relation $K(\omega)$. 

\begin{figure}
    %
    \includegraphics[width=1.0\textwidth, keepaspectratio]{figures/system/system.png}
    \caption{SQUID sites $n$ (blue) and CPW sections $n$ with flux field $\hat{\phi}_{\,n}(x,t)$ for 
    $n-1 < x < n$ (black). The matrix $S_n$ in Eq.\,(\ref{eq:matrix_s}) (red) transfers the flux field across the 
    boundary at site $n$. The matrix $P = P(1)$ in Eq.\,(\ref{eq:matrix_p2}) (green) propagates the flux field 
    through section $n+1$ from the right side of site $n$ to the left side of site $n+1$. Repeated application 
    of the transfer matrix $T_n = P(1) \, S_n$ in Eq.\,(\ref{eq:matrix_t2}) thus propagates the 
    flux field from the left side to the right side of the SQUID array.}
    \label{fig:system}
\end{figure}



%%%%%%%%%%%%%%%%%%%%%%%%%%%%%%%%%%%%%%%%%%%%%%%%%%%%%%%%%%%%%%%%%%%%%%%%%%%%%%%%%%%%%%%%%%%%%%%%%%%%%%%%%%%%%%

\section{Dynamical Casimir radiation in the periodic SQUID array}
\label{sec:dcr}

\noindent
We consider a linear, periodic array of ${\cal N} + 1$ SQUIDs $n = 0, \ldots , {\cal N}$
along the $x$-axis separated by a distance (lattice constant) $\ell$ and connected to 
coplanar waveguide (CPW) lines. In this section we use again unitless variables 
where the lattice constant is 1 and $x_n = n$.
The first SQUID (on the far left) is located at $x=0$ and the last one 
(on the far right) at $x = {\cal N}$ (see Figure \ref{fig:system}).

If the transfer matrix in Eq.\,(\ref{eq:matrix_t2}) is independent of the SQUID site $n$, 
i.e., $T_n = T$, by repeated application of Eq.\,(\ref{eq:matrix_t}) we can express the 
operators on the far right in terms of the operators on the far left:
%
\begin{equation} \label{eq:leftright}
\underline{X}_{\, \text{right}}(x={\cal N}) = U \underline{X}_{\,\text{left}}(x=0)
\end{equation}
%
with the $2M \times 2M$ matrix
%
\begin{equation} \label{eq:matrix_u}
U = S \, T^{\cal N} = S \, (P \, S)^{\cal N}
\end{equation}
%
and $\underline{X}_{\,\text{left}}$ and $\underline{X}_{\,\text{right}}$ are defined 
as in Eq.\,(\ref{eq:x_comp}) with vectors $\vec{X}_R$ and $\vec{X}_L$ as in 
Eq.\,(\ref{eq:vector_x}). 
In Section \ref{sec:results_symmetry_breaking} we will also consider an $n$-dependent harmonic drive 
$\delta \epsilon_n(t) = \delta \epsilon^0 \cos(\Omega \, t + \varphi_n)$
with $\delta \epsilon^0$ equal for all SQUIDs and
$\varphi_n = n (2 \pi) / 3$ (see Eq.\,(\ref{eq:deltaepsilon})).
This case is also described by Eqs.\,(\ref{eq:leftright}), (\ref{eq:matrix_u}) 
using a modified $U$-matrix for a periodically repeated unit 
cell containing 3 SQUIDs in an array containing a total number of $3 {\cal N} + 1$ SQUIDs:
%
\begin{equation} \label{eq:matrix_u2}
U' = S(\varphi=0) \, \left[T\left(\varphi =  \frac{4\pi}{3}  \right) T\left(\varphi =  \frac{2\pi}{3} \right) T(\varphi = 0) \right]^{\cal N} \, \, .
\end{equation}
%
\noindent
Using Eq.\,(\ref{eq:x_comp}) and the notation 
%
\begin{equation} \label{eq:abcd}
U \, = \,  
\begin{pmatrix}
a & b \\
c & d \\
\end{pmatrix}
\end{equation}
%
with $M \times M$ matrices $a$, $b$, $c$, $d$, 
Eq.\,(\ref{eq:leftright}) can be written as
%
\begin{equation} \label{eq:leftright2}
\begin{pmatrix}
\vec{X}_R \\
\vec{X}_L \\
\end{pmatrix}_{\, \text{right}} \, = \,
\begin{pmatrix}
a & b \\
c & d \\
\end{pmatrix}
\begin{pmatrix}
\vec{X}_R \\
\vec{X}_L \\
\end{pmatrix}_{\, \text{left}} \, \, .
\end{equation}
%
This equation relates a signal on the right of the SQUID array (incoming and outgoing)
to a signal on the left (compare Figure \ref{fig:xy}).
However, we would like to study a different situation: For given
{\em incoming} signals $\vec{X}_{R\,,\,\text{left}}$ and $\vec{X}_{L\,,\,\text{right}}$ 
(e.g., thermal noise) we would like to obtain the {\em outgoing} signals 
$\vec{X}_{R\,,\,\text{right}}$ and $\vec{X}_{L\,,\,\text{left}}$ generated or influenced 
by the dynamical DCE effect in the SQUID array.
This result can be obtained from Eq.\,(\ref{eq:leftright2}) by simple algebra, 
and we find
%
\begin{equation} \label{eq:inout}
\begin{pmatrix}
\vec{X}_{R\,,\,\text{right}} \\
\vec{X}_{L\,,\,\text{left}} \\
\end{pmatrix} \, = \,
\begin{pmatrix}
A & B \\
C & D \\
\end{pmatrix}
\begin{pmatrix}
\vec{X}_{R\,,\,\text{left}} \\
\vec{X}_{L\,,\,\text{right}} \\
\end{pmatrix} \, \, ,
\end{equation}
%
with $M \times M$ matrices
%
\begin{equation} \label{eq:bigabcd}
A = a - b \, d^{-1} c \, \, , \quad B = b \, d^{-1}  \, \, , \quad
C = - d^{-1} c \, \, , \quad D = d^{-1} \, \, ,
\end{equation}
%
where $d^{-1}$ is the matrix inverse of $d$. 
The transformation from Eq.\,(\ref{eq:leftright2}) to (\ref{eq:inout}) is 
well-defined since the matrix in Eq.\,(\ref{eq:leftright2}) is regular
due to the fact that all frequencies $\omega_{\,\alpha}$ contained in the 
$\vec{X}$-vectors are allowed 
by definition (see Eq.\,(\ref{eq:omega_range}) and the related discussion). 

Explicitly, Eq.\,(\ref{eq:inout}) reads for given frequency $\omega_{\,\alpha} > 0$, 
using the definition (\ref{eq:vector_x}),
%
\begin{equation} \label{eq:abcdexplicit}
\hat{a}_{\,\text{right}}^{\,R}(\omega_{\,\alpha}) \, = \, 
\sum_{\beta=1}^M A_{\alpha \beta}  \left. \vec{X}_{R\,,\,\text{left}} \right|_{\beta} \, + \,
\sum_{\beta=1}^M B_{\alpha \beta}  \left. \vec{X}_{L\,,\,\text{right}} \right|_{\beta}
\end{equation}
%
where $\displaystyle{\left. \vec{X}_{R\,,\,\text{left}} \right|_{\beta}}$ is the 
component $\beta$ of the vector $\displaystyle{\vec{X}_{R\,,\,\text{left\,}}}$. Equation \ref{eq:abcdexplicit}, gives the Bogoliubov transformations for our particular system (see Section \ref{sec:ZPE_and_DCE}).
Multiplying this equation from the left by $\hat{a}_{\,\text{right}}^{\,R\,\dagger}(\omega_{\,\alpha})$
we obtain the number operator for outgoing photons with frequency $\omega_{\,\alpha}$ 
moving to the right (R) on the right side (right) of the SQUID array:
%
\begin{equation} \label{eq:numberop} 
\begin{split}
& \hat{n}_{\,\text{right}}^{\,R}(\omega_{\,\alpha}) \, \equiv \, \, 
\hat{a}_{\,\text{right}}^{\,R\,\dagger}(\omega_{\,\alpha}) \, 
\hat{a}_{\,\text{right}}^{\,R}(\omega_{\,\alpha}) \\[3mm]
& = \, \sum\limits_{\gamma, \, \beta} A_{\alpha \gamma}^* \, \, A_{\alpha \beta}  
\left( \left. \vec{X}_{R\,,\,\text{left}} \right|_{\gamma} \right)^{\dagger}
\left. \vec{X}_{R\,,\,\text{left}} \right|_{\beta} \, + \,
\, \sum\limits_{\gamma, \, \beta} B_{\alpha \gamma}^* \, \, B_{\alpha \beta}  
\left( \left. \vec{X}_{L\,,\,\text{right}} \right|_{\gamma} \right)^{\dagger}
\left. \vec{X}_{L\,,\,\text{right}} \right|_{\beta} \\[3mm]
& \quad + \, \text{terms containing products} \, \, \, 
\hat{a}_{\,\text{left}}^{\dagger} \, \hat{a}_{\,\text{right}} \, \, , \, \, \, 
\hat{a}_{\,\text{right}}^{\dagger} \, \hat{a}_{\,\text{left}}
\, \, \, .
\end{split}
\end{equation}
%
\newpage
To find the DCE radiation associated with the photon number operator 
$\hat{n}_{\,\text{right}}^{\,R}(\omega_{\,\alpha})$ 
we define the density operator for a thermal equilibrium ensemble at temperature $T$
(statistical operator):
%
\begin{equation} \label{eq:do}
\hat{\rho}(T) \, = \, \frac{\exp\left[ - \hat{H} / (k_B T) \right]}{  \text{tr} \left\{ \exp\left[ - \hat{H} / (k_B T) \right] \right\}}
\end{equation}
%
where $k_B$ is the Boltzmann constant, tr denotes a trace, and the Hamilton operator 
in the basis of the incoming, non-interacting photon states 
on both sides of the array is given by
%
\begin{equation} \label{eq:hamilton}
\hat{H} \, = \, \int_{0}^{\infty} \frac{d{\omega}}{2 \pi} \, \, \hbar \omega \, 
\left[
\hat{a}_{\,\text{right}}^{\,L\,\dagger}(\omega) \, 
\hat{a}_{\,\text{right}}^{\,L}(\omega) \, + \,
\hat{a}_{\,\text{left}}^{\,R\,\dagger}(\omega) \, 
\hat{a}_{\,\text{left}}^{\,R}(\omega) 
\right] \, \, .
\end{equation}
%
Using Eq.\,(\ref{eq:numberop}) and $\hat{\rho}(T)$ from Eq.\,(\ref{eq:do}) we find 
the expectation value of the photon number of the outgoing mode with frequency 
$\omega_{\,\alpha}$ moving to the right (R) on the right side (right) of the SQUID array:
%
\begin{equation} \label{eq:numberthermal} 
\begin{split}
& n_{\,\text{right}}^{\,R}(\omega_{\,\alpha}, T) \, = \, 
\text{tr} \left\{ \hat{\rho}(T) \, \hat{n}_{\,\text{right}}^{\,R}(\omega_{\,\alpha}) \right\} \\[3mm]
& = \, \sum\limits_{\gamma, \, \beta} A_{\alpha \gamma}^* \, \, A_{\alpha \beta}  \, \,
\text{tr}  \left\{ \hat{\rho}(T) 
\left( \left. \vec{X}_{R\,,\,\text{left}} \right|_{\gamma} \right)^{\dagger}
\left. \vec{X}_{R\,,\,\text{left}} \right|_{\beta} \right\} \\[3mm]
& \, + \, \sum\limits_{\gamma, \, \beta} B_{\alpha \gamma}^* \, \, B_{\alpha \beta}  \, \, 
\text{tr}  \left\{ \hat{\rho}(T) 
\left( \left. \vec{X}_{L\,,\,\text{right}} \right|_{\gamma} \right)^{\dagger}
\left. \vec{X}_{L\,,\,\text{right}} \right|_{\beta} \right\}
\, \, \, .
\end{split}
\end{equation}
% 
The traces on the r.h.s.~of the above equation 
only contribute for $\gamma = \beta$ since the Hamilton operator $\hat{H}$ in Eq.\,(\ref{eq:hamilton}) 
is diagonal in $\omega$. 
Moreover, the terms containing products $\hat{a}_{\,\text{left}}^{\dagger} \, \hat{a}_{\,\text{right}}$ and 
$\hat{a}_{\,\text{right}}^{\dagger} \, \hat{a}_{\,\text{left}}$ in Eq.\,(\ref{eq:numberop})
vanish when taking the trace since the thermal noise on the left and right side of the array
is uncorrelated. We thus find, using the form of the vector 
$\vec{X}$ in Eq.\,(\ref{eq:vector_x}),

%
\begin{equation} \label{eq:numberthermal2} 
\begin{split}
& n_{\,\text{right}}^{\,R}(\omega_{\,\alpha}, T)  \\[3mm]
& = \, \sum\limits_{\beta=1}^{\kappa-1} \left|A_{\alpha \beta}\right|^2 \, \,
\text{tr} \left\{ \hat{\rho}(T) \,
\hat{a}_{\,\text{left}}^{\,R\,\dagger}(\omega_{\,\beta}) \, 
\hat{a}_{\,\text{left}}^{\,R}(\omega_{\,\beta})  
\right\} 
\, + \,
\sum\limits_{\beta=\kappa}^{M} \left|A_{\alpha \beta}\right|^2 \, \,
\text{tr} \left\{ \hat{\rho}(T) \,
\hat{a}_{\,\text{left}}^{\,R}(|\omega_{\,\beta}|) \, 
\hat{a}_{\,\text{left}}^{\,R\,\dagger}(|\omega_{\,\beta}|)  
\right\} \\[3mm]
& \, + \, \sum\limits_{\beta=1}^{\kappa-1} \left|B_{\alpha \beta}\right|^2 \, \,
\text{tr} \left\{ \hat{\rho}(T) \,
\hat{a}_{\,\text{right}}^{\,L\,\dagger}(\omega_{\,\beta}) \, 
\hat{a}_{\,\text{right}}^{\,L}(\omega_{\,\beta})  
\right\} 
\, + \,
\sum\limits_{\beta=\kappa}^{M} \left|B_{\alpha \beta}\right|^2 \, \,
\text{tr} \left\{ \hat{\rho}(T) \,
\hat{a}_{\,\text{right}}^{\,L}(|\omega_{\,\beta}|) \, 
\hat{a}_{\,\text{right}}^{\,L\,\dagger}(|\omega_{\,\beta}|)  
\right\}
\, \, \, .
\end{split}
\end{equation}
%
The traces in the first and third terms on the r.h.s.~of the above equation
are the occupation numbers of the incoming photons with frequencies $\omega_{\,\beta}>0$
on both sides of the array at thermal equilibrium (thermal noise):
%
\begin{equation} \label{eq:ton}
\begin{split}
\left. n(\omega_{\,\beta}, T) \right|_{\text{eq}} \, & = \, 
        \text{tr} \left\{ \hat{\rho}(T) \,
\hat{a}_{\,\text{left}}^{\,R\,\dagger}(\omega_{\,\beta}) \, 
\hat{a}_{\,\text{left}}^{\,R}(\omega_{\,\beta})  \right\} \\[4mm]
& = \, \text{tr} \left\{ \hat{\rho}(T) \,
\hat{a}_{\,\text{right}}^{\,L\,\dagger}(\omega_{\,\beta}) \, 
\hat{a}_{\,\text{right}}^{\,L}(\omega_{\,\beta})  \right\} \\[3mm]
& = \, \displaystyle{\frac{1}{\exp\left[ \hbar \omega_{\,\beta} / (k_B T) \right] - 1}} \, \, \, .
\end{split}
\end{equation} 
%
The traces in the second and fourth terms on the r.h.s.~in Eq.\,(\ref{eq:numberthermal2})
can also be expressed in terms of $\left. n(\omega_{\,\beta}, T) \right|_{\text{eq}}$
after using the commutation relation for a given frequency $\omega > 0$:
%
\begin{equation} \label{eq:c5}
\left[ \hat{a}(\omega), \hat{a}^{\dagger}(\omega) \right] \, = \, 1 \, \, \, \Rightarrow \, \, \, 
\hat{a}(\omega) \, \hat{a}^{\dagger}(\omega) \, = \, \hat{a}^{\dagger}(\omega) \, \hat{a}(\omega) \, + \, 1 \, \, .
\end{equation}
%
For example, for the second term on the r.h.s.\,in Eq.\,(\ref{eq:numberthermal2}) this gives
%
\begin{equation} \label{eq:c6}
\begin{split}
& \, \, \, \text{tr} \left\{ \hat{\rho}(T) \,
\hat{a}_{\,\text{left}}^{\,R}(|\omega_{\,\beta}|) \, 
\hat{a}_{\,\text{left}}^{\,R\,\dagger}(|\omega_{\,\beta}|) \right\} \\[3mm]
= & \, \, \, \text{tr} \left\{ \hat{\rho}(T) \,
\hat{a}_{\,\text{left}}^{\,R\,\dagger}(|\omega_{\,\beta}|) \, 
\hat{a}_{\,\text{left}}^{\,R}(|\omega_{\,\beta}|) \right\} \, + \, 
\text{tr} \left\{ \hat{\rho}(T) \right\} \\[3mm]
= & \, \, \left. n(|\omega_{\,\beta}|, T) \right|_{\text{eq}} \, + \, 1 \, \, .
\end{split}
\end{equation}
%
It is this operation, the application of the commutation relation for negative frequencies
$\omega_{\,\beta} < 0$ for $\beta \ge \kappa$ as in Eq.\,(\ref{eq:c6}), that results in the DCE 
radiation by the extra 1 on the r.h.s.\,of Eq.\,(\ref{eq:c6}) (see Eq.\,(\ref{eq:table_kappa})
and related discussion, and Figure \ref{fig:jump} on page \pageref{fig:jump}).
Using Eq.\,(\ref{eq:c6}) in Eq.\,(\ref{eq:numberthermal2}) we obtain the final result for the
expectation value of the photon number of the outgoing mode with frequency 
$\omega_{\,\alpha}$ moving to the right (R) on the right side (right) of the SQUID array:
%
\begin{equation} \label{eq:dce_right}
\begin{split}
n_{\,\text{right}}^{\,R}(\omega_{\,\alpha}, T) \, & = \,
\underbrace{\sum\limits_{\beta=1}^{M} \left|A_{\alpha \beta}\right|^2 \, \,
\left. n(|\omega_{\,\beta}|, T) \right|_{\text{eq}}}_{\text{thermal noise from left side}}
\, \, + \, \,
\underbrace{\sum\limits_{\beta=\kappa}^{M} \left|A_{\alpha \beta}\right|^2}_{\text{DCE radiation}} \\[4mm]
& + \, \underbrace{\sum\limits_{\beta=1}^{M} \left|B_{\alpha \beta}\right|^2 \, \,
\left. n(|\omega_{\,\beta}|, T) \right|_{\text{eq}}}_{\text{thermal noise from right side}}
\, \, + \, \,
\underbrace{\sum\limits_{\beta=\kappa}^{M} \left|B_{\alpha \beta}\right|^2}_{\text{DCE radiation}} \, \, ,
\end{split}
\end{equation}
%
with the matrices $A$, $B$ given by Eq.\,(\ref{eq:bigabcd}) and 
$\left. n(|\omega_{\,\beta}|, T) \right|_{\text{eq}}$ by Eq.\,(\ref{eq:ton}). 
In the same way we obtain the
expectation value of the photon number of the outgoing mode with frequency 
$\omega_{\,\alpha}$ moving to the left (L) on the left side (left) of the SQUID array:
%
\begin{equation} \label{eq:dce_left}
\begin{split}
n_{\,\text{left}}^{\,L}(\omega_{\,\alpha}, T) \, & = \,
\underbrace{\sum\limits_{\beta=1}^{M} \left|C_{\alpha \beta}\right|^2 \, \,
\left. n(|\omega_{\,\beta}|, T) \right|_{\text{eq}}}_{\text{thermal noise from left side}}
\, \, + \, \,
\underbrace{\sum\limits_{\beta=\kappa}^{M} \left|C_{\alpha \beta}\right|^2}_{\text{DCE radiation}} \\[4mm]
& + \, \underbrace{\sum\limits_{\beta=1}^{M} \left|D_{\alpha \beta}\right|^2 \, \,
\left. n(|\omega_{\,\beta}|, T) \right|_{\text{eq}}}_{\text{thermal noise from right side}}
\, \, + \, \,
\underbrace{\sum\limits_{\beta=\kappa}^{M} \left|D_{\alpha \beta}\right|^2}_{\text{DCE radiation}}
\end{split}
\end{equation}
%
with the matrices $C$, $D$ given by Eq.\,(\ref{eq:bigabcd}). 


%%%%%%%%%%%%%%%%%%%%%%%%%%%%%%%%%%%%%%%%%%%%%%%%%%%%%%%%%%%%%%%%%%%%%%%%%%%%%%%%%%%%%%%%%%%%%%%%%%%%%%%%%%%%%%

\section{Summary of computational procedure}
\label{sec:summary}

\noindent
In this section we summarize the procedure of calculating the photon number 
$n_{\,\text{right}}^{\,R}(\omega_{\,\alpha}, T)$ and 
$n_{\,\text{left}}^{\,R}(\omega_{\,\alpha}, T)$ 
in Eqs.\,(\ref{eq:dce_right}), (\ref{eq:dce_left})
for given system parameters. 
We use again unitless variables 
by expressing lengths in units of $\ell$
and times in units of $\ell/v$ where $\ell$ is the distance between SQUIDs in the periodic array
and $v$ is the speed of light in the CPW (see text after Eq.\,(\ref{eq:BC_freq_static})). 
The calculation of the band structure $\omega_{\,\nu}(k)$
and Bloch functions $\psi_{\nu,\,k}(x) = e^{i k x} u_{\nu,\,k}(x)$ for the 
Kronig-Penney type model discussed in Section \ref{subsec:kpsol} is 
illustrated in the Mathematica notebook in the Appendix. Results in Chapter \ref{ch:results} were generated using the full-length calculation with a Python program available in the author's public GitHub Repository \cite{Github_Repository}.
%
Using unitless variables, the only system-specific input parameters are:
%
\begin{itemize}
%
\item A given (unitless) frequency 
$\omega = \omega_{\,\alpha} > 0$ for which we want to calculate 
$n_{\,\text{right}}^{\,R}(\omega_{\,\alpha}, T)$ and 
$n_{\,\text{left}}^{\,R}(\omega_{\,\alpha}, T)$;
%
\item The energy parameter $\epsilon$ in Eqs.\,(\ref{eq:epsilon2}) and (\ref{eq:energyexp_alt});
%
\item The drive frequency $\Omega$, amplitudes $\delta \epsilon_n^0$ and 
phases $\varphi_n$ in the harmonic drive term 
$\delta \epsilon_n(t) = \delta \epsilon_n^0 \, \cos(\Omega \, t + \varphi_n)$
in Eq.\,(\ref{eq:energyexp_alt});

\item The number of repeated units ${\cal N}$ of the SQUID array;

\item The temperature $T$ in the thermal occupation numbers 
$\left. n(\omega_{\,\beta}, T) \right|_{\text{eq}}$ in Eq.\,(\ref{eq:ton}).
%
\end{itemize}

\noindent
The calculation of 
$n_{\,\text{right}}^{\,R}(\omega_{\,\alpha}, T)$ and 
$n_{\,\text{left}}^{\,R}(\omega_{\,\alpha}, T)$ proceeds as follows. 

\begin{enumerate}

\item \label{item:kp} 
Using unitless variables,
the band structure for the static case only depends on the parameter $\epsilon$
(see Section \ref{subsec:kpsol}).
Program the inverse dispersion relation $k(\omega, \epsilon) \in [0, \pi]$ 
in Eq.\,(\ref{eq:disp_rel_k}). 
If $\omega$ is allowed, the function should return the value $k(\omega)$. If $\omega$ is not allowed 
(i.e., falls in a frequency gap) the function should return a flag.
Extend this function to negative frequencies by defining $k(\omega) := k(|\omega|)$ for $\omega < 0$.
The function $k(\omega, \epsilon) > 0$ corresponds to $K(\omega)$ defined in Eq.\,(\ref{eq:kdef}).

\item \label{item:u}
Program a function $u(x, \omega, \epsilon)$ corresponding to 
$u_{\,\omega}(x) = u_{\nu, \, k(\omega)}(x)$ in Eq.\,(\ref{eq:psi1}).
A closed-form expression of $u(x, \omega, \epsilon)$ is given in the Mathematica notebook in the Appendix. 
%
where the function is called "u1" for $k>0$ (right-moving mode) and "u2" for $k<0$ (left-moving mode). 
In the notebook, $\omega$ is denoted $q$
(see footnote \ref{footnote:q} on page \pageref{footnote:q}) and the function 
$u_{\,\omega}(x)$ depends explicitly on $q = \omega$ and the function 
$k(\omega,\epsilon)$ of step \ref{item:kp}.
        
\item  \label{item:domain}
For the given frequency $\omega$ and drive frequency $\Omega$ identity the largest simply connected 
frequency domain that includes $\omega$ according to Eq.\,(\ref{eq:table}) using the function $k(\omega,\epsilon)$
of step \ref{item:kp}. This results in numbers
$N_u$, $N_l$, $M$ and a vector $\omega_{\,\alpha}$ with $\alpha = 1, \ldots, M$ as in 
Eq.\,(\ref{eq:omega_range}), where $\omega = \omega_{\,\alpha}$ with $\alpha = N_u + 1$.
(Some of these frequencies $\omega_{\,\alpha}$ may be negative.)

\item \label{item:kappa} 
Identify the index $\kappa$ according to Eq.\,(\ref{eq:table_kappa}). 
If all $\omega_{\,\alpha} > 0$ and no such $\kappa$ exists, then there is no DCE radiation with frequency $\omega$.

\item For the positive frequencies $\omega_{\,\alpha}>0$ with $\alpha = 1, \ldots, \kappa-1$
found in step \ref{item:domain} 
calculate the numbers ${\cal C}_{\,\alpha} = {\cal C}\left[\omega(\alpha)\right]$ and 
${\cal A}_{\,\alpha} = {\cal A}\left[\omega(\alpha)\right]$ using Eqs.\,(\ref{eq:calomega}) and (\ref{eq:cal}).
For the negative frequencies $\omega_{\,\alpha}<0$ with $\alpha = \kappa, \ldots, M$, 
calculate ${\cal C}_{\,\alpha} = {\cal C}\left(|\omega(\alpha)|\right)$ and
${\cal A}_{\,\alpha} = {\cal A}\left(|\omega(\alpha)|\right)^*$
using Eq.\,(\ref{eq:calomega2}).
This results in vectors ${\cal C}_{\,\alpha}$ and ${\cal A}_{\,\alpha}$ 
with $\alpha = 1, \ldots, M$.

\item For the frequencies $\omega_{\,\alpha}$ found in step \ref{item:domain} calculate the numbers
$d\omega(\alpha) = \displaystyle{\left|\frac{dk}{d\omega_{\,\alpha}} \right|^{1/2}}$ defined in Eq.\,(\ref{eq:dwn}), 
using the function $k(\omega,\epsilon)$ of step \ref{item:kp}. 
This results in a vector $d\omega(\alpha)$ with $\alpha = 1, \ldots, M$. 

\item For given SQUID site $n$, calculate the matrices $G_{\alpha \beta}^{\,(n)}$ and $A_{\alpha \beta}$
in Eqs.\,(\ref{eq:matrix_gtilde}) and (\ref{eq:matrix_a}) using the vectors found above.  

\item For given SQUID site $n$ calculate the matrix $S$ using Eqs.\,(\ref{eq:matrix_s}) and (\ref{eq:matrix_G}).  

\item Calculate the matrix $P(1)$ in Eqs.\,(\ref{eq:matrix_p}) and (\ref{eq:matrix_p2})
using the function $k(\omega,\epsilon)$ of step \ref{item:kp} and $\kappa$
found in step \ref{item:kappa}. (Note that $K(\omega)$ in Eq.\,(\ref{eq:matrix_p}) is positive by definition.)

\item If the matrix $S$ (and thus $T$) is independent of the SQUID site $n$, calculate the matrix 
$U$ using Eq.\,(\ref{eq:matrix_u}) with the matrices $S$ and $P$ found above 
and the number ${\cal N}$ of repeated units of the array. Otherwise, modify the calculation of $U$
according to the given setup (See Chapter \ref{ch:results}).

\item \label{item:abcd}
Identify the submatrices $a$, $b$, $c$, $d$ of $U$ according to Eq.\,(\ref{eq:abcd}), and use
them to calculate the matrices $A$, $B$, $C$, $D$ in Eq.\,(\ref{eq:bigabcd}). 

\item \label{item:dce}
Use the matrices $A$, $B$, $C$, $D$ to calculate 
$n_{\,\text{right}}^{\,R}(\omega_{\,\alpha}, T)$ and 
$n_{\,\text{left}}^{\,R}(\omega_{\,\alpha}, T)$ 
in Eqs.\,(\ref{eq:dce_right}), (\ref{eq:dce_left}). 

\end{enumerate}

%%%%%%%%%%%%%%%%%%%%%%%%%%%%%%%%%%%%%%%%%%%%%%%%%%%%%%%%%%%%%%%%%%%%%%%%%%%%%%%%%%%%%%%%%%%%%%%%%%%%%%%%%%%%%%

\section{Special case: Single SQUID}
\label{sec:single}

\noindent
As a test case, we also consider a single SQUID in a CPW, 
corresponding to ${\cal N}=1$. 
In this section we summarize the required modifications to calculate the photon number 
$n_{\,\text{right}}^{\,R}(\omega_{\,\alpha}, T)$ and 
$n_{\,\text{left}}^{\,R}(\omega_{\,\alpha}, T)$ 
in Eqs.\,(\ref{eq:dce_right}), (\ref{eq:dce_left}) on both sides of the SQUID. 

\begin{itemize}

\item The Bloch modes $\psi_k(x)$ in Eq.\,(\ref{eq:flux_field_orig}) should be 
replaced by plane waves $e^{i k x}$, which amounts to setting $u_k(x)=1$ in Eq.\,(\ref{eq:psi2}). 
The dispersion relation is simply $\omega(k) = v |k|$ corresponding to $\omega(k) = |k|$
in unitless variables. Assuming that the SQUID is at $x=0$
Eq.\,(\ref{eq:flux_field_orig}) reduces to
%
\begin{equation} \label{eq:flux_field_single}
\begin{split}
    \hat{\Phi}_{\pm}(x,t) \, = & \, \sqrt{\frac{\hbar}{2 C_0}} \, 
    \int_{-\infty}^{\infty}\frac{dk}{2 \pi} \,
    \frac{1}{\sqrt{\omega(k)}} \\[3mm]
    & \times \, \left( \, \hat{a}_{\pm}(k) \exp\left[i k x - i \omega(k) \, t \right] \, + \, 
    \hat{a}_{\pm}^{\dagger}(k) \exp\left[- i k x + i \omega(k) \, t \right] \, \right) \, \, ,
\end{split}
\end{equation}
%
for $x>0$ ($+$) and $x<0$ ($-$). 

\item
Instead of Eq.\,(\ref{eq:defrl}), annihilation operators in frequency space 
are defined by
%
\begin{subequations} \label{eq:defrl_single}
\begin{eqnarray}
& \hat{a}_{\pm}^{\,R}(\omega) := \hat{a}_{\pm} \left[k(\omega) \right] \, \, , 
 \quad \text{moving to the right (R)} \, \, , \\[2mm]
& \hat{a}_{\pm}^{\,L}(\omega) := \hat{a}_{\pm} \left[ - k(\omega) \right] \, \, , 
 \quad \text{moving to the left (L)} \, \, ,
\end{eqnarray}
\end{subequations}
%
where $k(\omega) > 0$ as before, i.e., $k(\omega) = |\omega|$ (compare footnote \ref{foot:k} 
on page \pageref{foot:k}). 

\item
Equation (\ref{eq:flux_field2}) is replaced by 
%
\begin{equation} \label{eq:flux_field_single_unitless}
    \hat{\phi}_{\pm}(x,t) \, = \, 
    \displaystyle{
    \int_{-\infty}^{\infty} \, \frac{d\omega}{2 \pi} \,
        \frac{1}{\sqrt{|\omega|}}} \\[1mm]
    \times \left[ \, \hat{a}_{\pm}^{\,R}(\omega) e^{i \omega x} \, + \, 
      \hat{a}_{\pm}^{\,L}(\omega) e^{-i \omega x}  \, \right] e^{-i \omega \, t} \, \, ,
\end{equation}
%
where for $\omega < 0$ we define
$\hat{a}_{\pm}^{\,R}(\omega) = \hat{a}_{\pm}^{\,R\,\dagger}(|\omega|)$ and 
$\hat{a}_{\pm}^{\,L}(\omega) = \hat{a}_{\pm}^{\,L\,\dagger}(|\omega|)$.
Note that for $\omega < 0$ Eq.\,(\ref{eq:flux_field_single_unitless}) incorporates the correct signs
in the plane-wave terms since 
$e^{- i k(\omega) x} = e^{i \omega x}$ for $\omega < 0$
with $k(\omega) = |\omega|$ (compare Eqs.\,(\ref{eq:nf}) and (\ref{eq:nf2})). 

\item
Since for a single SQUID there are no frequency gaps, 
the numbers $N_u$ and $N_l$ in Eq.\,(\ref{eq:table}) are equal to the global cutoff, 
i.e., $N_u = N_l = N_{cut}$ and $M = 2 N_{cut} + 1$. 
Identify the index $\kappa$ according to Eq.\,(\ref{eq:table_kappa}) as before. 


\item Since $u(x) = 1$ in Eqs.\,(\ref{eq:cal}) - (\ref{eq:calomega2}) we obtain 
${\cal C}_{\omega} = 1$ and ${\cal A}_{\omega} = i \omega$ for all $\omega$
(incorporating the correct sign in ${\cal A}_{\omega}$ for $\omega<0$ 
since $(i \, |\omega|)^* = i \omega$ for $\omega < 0$).

\item Since $\omega(k) = |k|$ we obtain
$d\omega(\alpha) = \displaystyle{\left|\frac{dk}{d\omega_{\,\alpha}} \right|^{1/2}} = 1$
for all $\omega_{\,\alpha} > 0$.

\item According to the above, 
the matrices $\delta G_{\alpha \beta}^{\,(n)}$ and $A_{\alpha \beta}$
in Eqs.\,(\ref{eq:matrix_gtilde}) and (\ref{eq:matrix_a}) are replaced by
%
\begin{equation} \label{eq:matrix_gtilde_single}
G_{\alpha \beta} \, = \, \delta_{\alpha \beta} \, + \, \frac{1}{2} \frac{\delta \epsilon_n^0}{\epsilon} \, 
\sqrt{\frac{|\omega_{\alpha}|}{|\omega_{\beta}|}} \,
\left(\delta_{\,\beta,\,\alpha+1} + \delta_{\,\beta,\,\alpha-1} \right) \, \, ,
\end{equation}
%
\begin{equation} \label{eq:matrix_a_single}
A_{\alpha \beta} \, = \, \frac{i \omega_{\,\alpha}}{\epsilon} \, \delta_{\alpha \beta} \, \, .
\end{equation}
%
Thus, the matrix ${\cal G}$ in Eq.\,(\ref{eq:matrix_G}) is replaced by
%
\begin{equation} \label{eq:matrix_G_single}
{\cal G}_{\alpha \beta} \, = \, - i \, \frac{\epsilon}{2 \omega_{\,\alpha}} G_{\alpha \beta} \, \, .  
\end{equation}
%
The term $\delta_{\alpha \beta}$ in Eq.\,(\ref{eq:matrix_gtilde_single}) (first term on the r.h.s.) 
incorporates the contribution from the term proportional to $\epsilon$ in the boundary condition
(\ref{eq:BC_field_2}) when applied to plane waves $e^{i k x}$ at a single SQUID in the absence of 
a periodic lattice
%
\footnote{Note that for a periodic lattice, the contributions proportional to $\epsilon$ 
in the boundary condition at the SQUID sites $n$ are incorporated in the Bloch functions 
in Eq.\,(\ref{eq:psi1}), which can travel freely through the lattice in the static case; 
see Section \ref{subsec:kpsol}.}.

\item Calculate the matrix $S$ using Eqs.\,(\ref{eq:matrix_s}) and (\ref{eq:matrix_G})
as before, now using Eqs.\,(\ref{eq:matrix_gtilde_single}) and (\ref{eq:matrix_a_single}).
The matrix $P(1)$ in Eqs.\,(\ref{eq:matrix_p}) and (\ref{eq:matrix_p2}) is not needed.
Identify $U \equiv S$ and proceed as in steps \ref{item:abcd} and \ref{item:dce} in 
Section \ref{sec:summary}.

 \end{itemize}

%%%%%%%%%%%%%%%%%%%%%%%%%%%%%%%%%%%%%%%%%%%%%%%%%%%%%%%%%%%%%%%%%%%%%%%%%%%%%%%%%%%%%%%%%%%%%%%%%%%%%%%%%%%%%%